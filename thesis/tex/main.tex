% -----------------------------------
% -----------------------------------
% abnTeX2: Normas ABNT NBR 14724:2011 + sugestões FGV/EMAp. 

% Autor: Lauro César Araujo
% Adaptações EMAp: Lucas Machado Moschen 
% Copyright 2012-2018 by abnTeX2 group at http://www.abntex.net.br/ 

%% This work may be distributed and/or modified under the
%% conditions of the LaTeX Project Public License, either version 1.3
%% of this license or (at your option) any later version.
%% The latest version of this license is in
%%   http://www.latex-project.org/lppl.txt
%% and version 1.3 or later is part of all distributions of LaTeX
%% version 2005/12/01 or later.
% ----------------------------------
% ----------------------------------
\documentclass[
	% -- opções da classe memoir --
	12pt,				% tamanho da fonte
	%openright,			% capítulos começam em página ímpar (insere página vazia caso preciso)
	oneside,			%
	a4paper,			% tamanho do papel. 
	% -- opções da classe abntex2 --
	%chapter=TITLE,		% títulos de capítulos convertidos em letras maiúsculas
	%section=TITLE,		% títulos de seções convertidos em letras maiúsculas
	%subsection=TITLE,	% títulos de subseções convertidos em letras maiúsculas
	%subsubsection=TITLE,% títulos de subsubseções convertidos em letras maiúsculas
	% -- opções do pacote babel --
	% brazil,				% idioma para português
	english,			% idioma para inglês
	]{abntex2}

% if you need to pass options to natbib, use, e.g.:
%     \PassOptionsToPackage{numbers, compress}{natbib}
% before loading neurips_2023


% ready for submission
\usepackage[nonatbib]{neurips_2023}
\usepackage{global-macros}


% to compile a preprint version, e.g., for submission to arXiv, add add the
% [preprint] option:
%     \usepackage[preprint]{neurips_2023}

% to compile a camera-ready version, add the [final] option, e.g.:
%     \usepackage[final]{neurips_2023}


% to avoid loading the natbib package, add option nonatbib:
%    \usepackage[nonatbib]{neurips_2023}


% \usepackage[utf8]{inputenc} % allow utf-8 input
\usepackage[T1]{fontenc}    % use 8-bit T1 fonts
\newcommand{\theHalgorithm}{\arabic{algorithm}}
\usepackage[algoruled,boxed,lined]{algorithm2e}
\usepackage{hyperref}       % hyperlinks
\usepackage{url}            % simple URL typesetting
\usepackage{booktabs}       % professional-quality tables
\usepackage{amsfonts}       % blackboard math symbols
\usepackage{nicefrac}       % compact symbols for 1/2, etc.
\usepackage{microtype}      % microtypography
\usepackage{enumitem}
\usepackage[dvipsnames]{xcolor}         % colors
\usepackage{biblatex}
\addbibresource{../bib/bibliography.bib}

\usepackage{xargs}                      % Use more than one optional parameter in a new commands
% \usepackage[pdftex,dvipsnames]{xcolor}  % Coloured text etc.

\usepackage[colorinlistoftodos,prependcaption,textsize=tiny]{todonotes}
\newcommandx{\unsure}[2][1=]{\todo[linecolor=red,backgroundcolor=red!25,bordercolor=red,#1]{#2}}
\newcommandx{\change}[2][1=]{\todo[linecolor=blue,backgroundcolor=blue!25,bordercolor=blue,#1]{#2}}
\newcommandx{\info}[2][1=]{\todo[linecolor=OliveGreen,backgroundcolor=OliveGreen!25,bordercolor=OliveGreen,#1]{#2}}
\newcommandx{\improvement}[2][1=]{\todo[linecolor=Plum,backgroundcolor=Plum!25,bordercolor=Plum,#1]{#2}}
\newcommandx{\thiswillnotshow}[2][1=]{\todo[disable,#1]{#2}}


\title{Stochastic Gradient Descent in NPIV estimation}


% The \author macro works with any number of authors. There are two commands
% used to separate the names and addresses of multiple authors: \And and \AND.
%
% Using \And between authors leaves it to LaTeX to determine where to break the
% lines. Using \AND forces a line break at that point. So, if LaTeX puts 3 of 4
% authors names on the first line, and the last on the second line, try using
% \AND instead of \And before the third author name.


\author{%
}

\newcommand{\hstar}{h^{ \star }}
\newcommand{\meanop}{\mathcal{P}}
\newcommand{\op}{\mathrm{op}}
\newcommand{\risk}{\mathcal{R}}
\newcommand{\loss}{\ell}
\renewcommand{\hat}{\widehat}
\newcommand{\iid}{\overset{\mathrm{iid}}{\sim}}
\DeclareMathOperator{\diam}{diam}
\newcommand{\data}{\mathcal{D}}
\newcommand{\dataoff}{\mathcal{D}_{ \Phi, \meanop, r_{ 0 } }}
\newcommand{\dataproj}{\mathcal{D}_{ \mathrm{proj} }}

\newcommand{\rkhsw}{\mathcal{R}_{ \mathbb{W} }}

% Environments
\newtheorem{assumption}{Assumption}
\newtheorem{proposition}{Proposition}

\hfuzz=15pt



%-------------------------------------------------
%----------------- Document ----------------------
%-------------------------------------------------

\begin{document}

\newcounter{num}
% if num != 1, do not print the pre textual 
\setcounter{num}{1}

\selectlanguage{english}
\frenchspacing 

%----------------------------------------------
%--------------- Pré-textuais -----------------
%----------------------------------------------
%\pretextual

\imprimircapa

\ifnum\value{num}=1
{
    % Locally remove limits for underfull hbox so latex does
    % not file a warning.
    \hbadness=10000
    \hfuzz=17pt

    \imprimirfolhaderosto*

    \begin{fichacatalografica}
	\sffamily
	\vspace*{\fill}					% Posição vertical
	\begin{center}					
	\fbox{\begin{minipage}[c][8cm]{13.5cm}		% Largura
    \hfuzz=17pt
	\small
	Ficha catalográfica elaborada pela BMHS/FGV \\

	%\imprimirautor
	Lins, Caio % Paginas com as citações na bibl
	
	\hspace{0.5cm} \imprimirtitulo / \imprimirautor. -- \imprimirdata.
	
	\hspace{0.5cm} \thelastpage f.\\
		
	\hspace{0.5cm}
	\parbox[t]{\textwidth}{\imprimirtipotrabalho~--~School of Applied
	Mathematics.}\\
	
	\hspace{0.5cm} Advisor: \imprimirorientador .

	\hspace{0.5cm} Includes bibliography. \\
	
	\hspace{0.5cm}
		1. Inverse Problems
		2. Kernel Regression
		2. Instrumental Variable Regression
		I. Saporito, Yuri Fahham
		II. School of Applied Mathematics.
		III. \imprimirtitulo 			
	\end{minipage}}
	\end{center}
\end{fichacatalografica}
% Uncomment if you have the pdf 
% \begin{fichacatalografica}
%     \includepdf{fig_ficha_catalografica.pdf}
% \end{fichacatalografica}


    \begin{folhadeaprovacao}

    \begin{center}
      {\ABNTEXchapterfont\large\MakeUppercase{\imprimirautor}}

      \vspace*{\fill}\vspace*{\fill}
      \begin{center}
        \ABNTEXchapterfont\bfseries\large\MakeUppercase{\imprimirtitulo}
      \end{center}
      \vspace*{\fill}
    
      \hfill
      \begin{minipage}{.7\textwidth}
          \imprimirpreambulo \\ \\
          Approved on December ----, 2023 \\
          By the organizing committee
      \end{minipage}%
      \vspace*{\fill}
     \end{center}

     \assinatura{\imprimirorientador \\ School of Applied Mathematics} 
     \assinatura{Board Member 1 \\ Institution 1}
     \assinatura{Board Member 2 \\ Institution 2}
     %\assinatura{\textbf{Professor} \\ Convidado 3}
     %\assinatura{\textbf{Professor} \\ Convidado 4}
\end{folhadeaprovacao}

% \begin{folhadeaprovacao}
%   \includepdf{files/folha_aprovacao_nosign.pdf}
% \end{folhadeaprovacao}


    \newpage

\begin{dedicatoria}
    \vspace*{\fill}
    %\noindent
    \hfill
    \begin{minipage}{.6\textwidth}
        Para Helder e Mariza.
    \end{minipage}
\end{dedicatoria}
 
\begin{agradecimentos}
    In consideration for those mentioned here, this section will be written in my mother tongue, Portuguese.

    Este trabalho marca o final da minha graduação na EMAp e, portanto, gostaria de agradecer a todos aqueles que, de alguma forma, contribuíram para essa jornada.

    Agradeço, em primeiro lugar, a meus pais, Helder e Mariza, pelo apoio incondicional que sempre deram a mim.
    Sem o amor, a dedicação e o incentivo de vocês, eu jamais estaria onde estou hoje.
    
    Agradeço à minha namorada, Fran, por sempre estar ao meu lado e acreditar em mim, mesmo quando a distância parecia ser grande demais.
    Saber que você estaria me esperando em BH me dava forças para continuar seguindo em frente.

    Agradeço aos meus amigos da Bald Comics, Adame e Tulio, por serem minha companhia constante e por tornarem mais leves todos os perrengues da graduação.
    Sou muito sortudo de ter amigos tão talentosos e atenciosos quanto vocês.
    Aos meus amigos de mais longa data, também deixo meus mais sinceros agradecimentos: Gabriel, Elias, Tiago, Miguel e Motta.
    Mesmo distantes fisicamente, vocês se fizeram presentes em diversos momentos, e isso teve um valor inestimável.

    Agradeço ao Centro de Desenvolvimento da Matemática e Ciências (CDMC) da FGV, por ter me proporcionado a oportunidade de cursar minha graduação na EMAp, algo que mudou completamente o rumo da minha vida.
    Em particular, agradeço a Cássia Pessanha e Luziel Claret pelo suporte constante, e ao diretor do CDMC, César Camacho, por ter idealizado esse projeto e por atuar constantemente para mantê-lo viável.

    Agradeço a meu colaborador Yuri Resende, por todas as discussões proveitosas a respeito deste projeto.

    Por último, mas de longe não menos importante, agradeço ao meu orientador, Yuri Saporito.
    Foram, ao total, três anos em que tive o privilégio de poder tê-lo como professor, colaborador e amigo.
    Você me acompanhou desde quando eu não sabia o que era uma variável aleatória até agora, nunca deixou de acreditar no meu potencial e sempre me incentivou a buscar o que é melhor para mim.
    Por isso, sempre serei grato.
\end{agradecimentos}

\begin{epigrafe}
\vspace*{\fill}

\begin{flushright}
    \hspace{7.5cm}
    \textit{
        ``
        It was as though applied Mathematics was my spouse, and pure Mathematics was my secret lover.
        ''
    } \\
        \textit{Edward Frenkel}
\end{flushright}
\end{epigrafe}


    \hbadness=0
    \hfuzz=0pt

    \setlength{\absparsep}{18pt} 

\begin{resumo}[Abstract]
 \begin{otherlanguage*}{english}
     Instrumental variables (IVs) are a widely used tool in econometrics to address regression problems in which some covariates are correlated with the error term.
     In the nonparametric framework, one would like to find the true functional parameter from an infinite dimensional search space.
     This work provides an introduction to IV regression, both in its parametric and nonparametric forms, and presents a novel method to tackle the latter, based on a functional variant of stochastic first order methods.
     Theoretical as well as empirical results are presented.
 \end{otherlanguage*}

 Keywords: Nonparametric Regression, Instrumental Variables, Stochastic Optimization.
\end{resumo}

\begin{resumo}[Resumo]
    Variáveis instrumentais (VIs) são uma ferramenta amplamente utilizada em econometria para tratar problemas de regressão em que algumas covariáveis são correlacionadas com termo de erro.
    No âmbito não paramétrico, o objetivo é encontrar o verdadeiro parâmetro funcional em um espaço de procura de dimensão infinita.
    Este trabalho fornece uma introdução a regressão por VI, tanto no formato paramétrico quanto não paramétrico, e apresenta um novo método para tratar o último, baseado em uma variante funcional de métodos estocásticos de primeira ordem.
    Resultados teóricos e empíricos são apresentados.

    Palavras-chave: Regressão Não Paramética, Variáveis Instrumentais, Otimização Estocástica.
\end{resumo}


    \pdfbookmark[0]{\listfigurename}{lof}
    \listoffigures*
    \cleardoublepage

    % \pdfbookmark[0]{\listofquadrosname}{loq}
    % \listofquadros*
    % \cleardoublepage

    \pdfbookmark[0]{\listtablename}{lot}
    \listoftables*
    \cleardoublepage

}\fi

\pdfbookmark[0]{\contentsname}{toc}
\tableofcontents*
\cleardoublepage

% ----------------------------------------------------------
% ELEMENTOS TEXTUAIS
% ----------------------------------------------------------
\textual

\chapter{Introduction}

When choosing a method to tackle a regression problem $ Y = h_{ \theta } ( X ) + \varepsilon $, one is often concerned with certain properties of the resulting estimator $ \hat{ \theta } $, such as consistency and unbiasedness.
The widely employed linear model estimator, for example, enjoys these properties if the problem at hand satisfies its usual hypotheses, which include assuming that the noise term $ \varepsilon $ is uncorrelated with the covariate $ X $.
However, in practical scenarios it is very common to have external unmeasured factors influence both $ X $ and $ \varepsilon $, which introduces some correlation between these variables.
In these cases, where $ X $ is said to be \emph{endogenous}, methods based on ordinary least squares turn out to be biased and inconsistent.
To address this problem, econometric practitioners invented the method of instrumental variables, which uses a third measurable quantity, $ Z $, that influences $ X $ but is uncorrelated with $ \varepsilon $, to correct the faults in ordinary least squares estimators.

This work deals with the nonparametric version of this problem, where $ h_{ \theta } $ is a member of an infinite dimensional space.
In recent years we have seen a ressurgence of interest in nonparametric instrumental variable (NPIV) regression, with multiple methods being devised and published, such as the ones in \cite{singh2019}, \cite{deepiv2017}, \cite{deepgmm2019} and \cite{dualiv2020}.

About the structure of this document, in Chapter 2 we provide an overview of instrumental variables in econometric practice and present the Two Stages Least Squares estimator, which generalizes the linear model to the case where some covariates are endogenous.
In Chapter 3 we show how NPIV regression can be seen as an ill-posed linear inverse problem and present two well established methods to tackle it.
Chapter 4 then derives and analyses SAGD-IV, our own estimator for NPIV regression problems based on a functional variant of stochastic first order methods.
A convergence speed bound is shown, as well as an empirical performance check using artificial data.
Our method stands out because it can increase the quality of the estimator only using additional samples from $ Z $, and our theoretical results, albeit not as strong, are proved under weak assumptions when compared to the literature.


% ----------------------------------------------------------
% Finaliza a parte no bookmark do PDF
% para que se inicie o bookmark na raiz
% e adiciona espaço de parte no Sumário
% ----------------------------------------------------------
\phantompart

\chapter{Application in Econometrics: Instrumental Variable Regression}

In this chapter, we turn our attention back to the main application being considered in this thesis: instrumental variable  regression.
We start by characterizing the type of problem this econometric approach was developed to solve, and then present one of the most well-known and employed\unsure{Is it?} estimation procedures for conducting it: two stages least squares (2SLS).
Nextly, we show how an IV regression problem can be formulated as a linear inverse problem and discuss the seminal nonparametric method of Newey and Powell \cite{newey2003}, followed by a more recent nonparametric approach called Kernel Instrumental Variable (KIV) \cite{singh2019}.
We finish with an in-depth analysis of our own method\improvement{Name of the method.}, pointing out its relationship to the others, as well as its strengths and weaknesses.

\section{The Role of Instrumental Variables}

Our first goal is to introduce the problem of endogenous covariates.
The structural equation we consider is the following:
\begin{equation}
    \label{eq: structural equation}
    Y = \hstar ( X ) + \varepsilon
,\end{equation}
where $ X $ is a $ d $-dimensional vector of explanatory variables, $ Y $ is the scalar response, $ \varepsilon $ is a zero mean noise and the function $ \hstar $ is the structural parameter we would like to estimate.
% In what follows, we restrict ourselves to linear models, and thus assume $ \hstar $ has the following parametric form:
% \begin{equation}
%     \label{eq: hstar is affine}
%     \hstar ( \bx ) = \beta_{ 0 } + \beta_{ 1 } \bx_{ 1 } + \cdots + \beta_{ d } \bx_{ d } = \beta^{ \trp } \bx
% .\end{equation}
The simplest estimation method for this model specification --- and, therefore, one we would like to be able to use --- is ordinary least squares (OLS), which works by finding, within a given class of functions $ \mathcal{H} $, the element which minimizes the mean squared error:
\begin{equation}
    \label{eq: ols estimate}
    \hat{ h } = \argmin_{ h \in \mathcal{H} } \mean [ 
        ( Y - h ( X ) )^2
    ]
.\end{equation}
A reasonable and ample choice for $ \mathcal{H} $ is the set of all square-integrable functions of $ X $, that is, such that $ \mean [ h ( X )^2 ] < \infty $.
Under this choice, we recover the conditional expectation of $ Y $ given $ X $, i.e., $ \hat{ h } ( X ) = \mean [ Y \mid X ] $.
Expanding $ Y $ through (\ref{eq: structural equation}), we find that $ \hat{ h } ( X ) = \hstar ( X ) + \mean [ \varepsilon \mid X ] $.
Hence, if $ \mean [ \varepsilon \mid X ] $ is not identically null, we have introduced bias in our estimation.

This is one of the problems which appear when $ \mean [ \varepsilon \mid X ] \neq 0 $, or, more generally, when $ X $ and $ \varepsilon $ are correlated in some way.
When this happens, we say that $ X $ i\emph{endogenous}.
There are several causes for endogenous covariates, the most common of  which are \cite{wooldridge2001}\unsure{Provide one example for each}:
\begin{description}
    \item[Omitted Variables] This means $ \varepsilon $ can be decomposed as $ \gstar ( W ) + \eta $, where $ \mean [ \eta \mid X, W ] = 0 $ a.s. and $ X $ and $ W $ are correlated.
        Hence, when we don't observe $ W $ and leave it to the error term, we end up estimating $ \hstar ( X ) + \mean [ \gstar ( W ) \mid X ] $.
    \item[Measurement Error] If we are unable to exactly measure one of the covariates, $ X_{ k } $, and instead measure $ X_{ k }' $ subject to some stochastic error, by using $ X_{ k }' $ in our regression instead of $ X_{ k } $ we are delegating to $ \varepsilon $ some measure of the difference between $ X_{ k } $ and $ X_{ k }' $.
        Depending on how these two variables are related, we may introduce endogeneity.
    \item[Simultaneity] Simultaneity arises when one covariate $ X_{ k } $ is determined simultaneously with $ Y $.
        For example, if we are regressing neighborhood murder rates using the size of the local task force as a covariate, there is a simultaneity problem, since larger murder rates in a place cause a larger task force to be allocated there.
\end{description}

Bias in the estimation procedure is only one of the problems which arise when there are endogenous covariates.
It's well known that the OLS estimate for linear regression fails to be consistent if any one of the covariates is endogenous \cite{wooldridge2001}.
To overcome endogeneity a few approaches exist, but by far the one most used by empirical economic research is instrumental variable estimation \cite{wooldridge2001}.


\chapter{Nonparametric Instrumental Variable Regression and Linear Inverse Problems}


\phantompart

\chapter{Conclusion}
\label{conclusion}

In this thesis, we have presented an overview of instrumental variable regression, dived into more details about some relevant approaches which have been made to the nonparametric version and explained our own attempt to tackle this problem.

In our future work, we aim at providing more robust benchmarks of SAGD-IV against other popular algorithms, as well as with more challenging data generating processes.
We also wish to test its performance in cases where the response is binary, i.e., when we have $ Y = \ind \left\{ \hstar ( X ) + \varepsilon > 0 \right\} $, since our theoretical results are applicable in this situation.







% -----------------------------------
% ELEMENTOS PÓS-TEXTUAIS
% -----------------------------------
\postextual
% ----------------------------------

%\bibliography{biblio}
\printbibliography

%\glossary

% ----------------------------------------------------------
% Apêndices
% ----------------------------------------------------------

\begin{apendicesenv}

\partapendices

\chapter{Proofs of theoretical results}

\section{Proof of Theorem \ref{thm: main theorem}}

To lighten the notation, the symbols $ \norm{ \cdot } $ and $ \dotprod{ \cdot, \cdot } $, when written without a subscript to specify which space they refer to, will act as the norm and inner product, respectively, of $ L^2 ( X ) $.
Before presenting the proof of Theorem \ref{thm: main theorem}, we need to prove two auxiliary lemmas:
\begin{lemm}
    \label{lem: bound u_m}
    In the procedure of Algorithm \ref{algo: sagdiv} we have $ u_{ m } \in L^{ 2 } ( X ) $ for all $ 1 \leq m \leq M $ and, furthermore,
    \begin{equation*}
        \mean_{ \bz_{ 1:M } } [ \norm{ u_{ m } }^2 ] \leq
        \rho \left( \hat{ \Phi }, \hat{ r }, \hat{ \meanop } \right) 
    ,\end{equation*}
    where
    \begin{equation*}
        \rho \left( \hat{ \Phi }, \hat{ r }, \hat{ \meanop } \right) =
        3 \norm{ \hat{ \Phi } }_{ \infty }^2 \left(
            C_{ 0 }^2 + L^2 \norm{ \hat{ r } }_{ L^2 ( Z ) }^2 + L^2 D^2 \norm{ \hat{ \meanop } }_{ \op }^2
        \right)
    .\end{equation*}
\end{lemm}
\begin{proof}
    By Assumption \ref{assumption estimators} we have:
    \begin{align*}
        \norm{ u_{ m } }_{ L^{ 2 } ( X ) }^2
        &= \norm{
            \hat{ \Phi } ( \cdot, \bz_{ m } ) \partial_{ 2 } \ell \left(
                \hat{ r } ( \bz_{ m } ), \hat{ \meanop } [ \hat{ h }_{ m-1 } ] ( \bz_{ m } )
            \right)
        }_{ L^{ 2 } ( X ) }^2  \\
        &= \mean_{ X } \left[
            \abs{ 
                \hat{ \Phi } ( X, \bz_{ m } ) \partial_{ 2 } \ell \left(
                    \hat{ r } ( \bz_{ m } ),
                    \hat{ \meanop } [ \hat{ h }_{ m-1 } ] ( \bz_{ m } )
                \right)
            }^2
        \right]  \\
        &\leq \partial_{ 2 } \ell \left(
            \hat{ r } ( \bz_{ m } ),
            \hat{ \meanop } [ \hat{ h }_{ m-1 } ] ( \bz_{ m } )
        \right)^2
        \norm{ \hat{ \Phi } }_{ \infty }^2  \\
        &< \infty 
    .\end{align*}
    Hence, $ u_{ m } \in L^{ 2 } ( X ) $ for all $ m $.
    This computation and Proposition \ref{prop: loss properties} \ref{bounded growth} then imply
    \begin{align*}
        \mean_{ \bz_{ 1:M } } \left[
            \norm{ u_{ m } }^2
        \right]
        &\leq 3 \norm{ \hat{ \Phi } }^2_{ \infty } 
        \left(
            C_{ 0 }^2 + L^2 \left(
                \norm{ \hat{ r } }_{ L^2 ( Z ) }^2
                + \norm{ \hat{ \meanop } [ \hat{ h }_{ m-1 } ] }_{ L^2 ( Z ) }^2
            \right)
        \right) \\
        &\leq 3 \norm{ \hat{ \Phi } }^2_{ \infty }
        \left(
            C_{ 0 }^2 + L^2 \left(
                \norm{ \hat{ r } }_{ L^2 ( Z ) }^2
                + \norm{ \hat{ \meanop } }_{ \op }^2 \norm{ \hat{ h }_{ m-1 } }^2
            \right)
        \right) \\
        &\leq 3 \norm{ \hat{ \Phi } }^2_{ \infty }
        \left(
            C_{ 0 }^2 + L^2 \left(
                \norm{ \hat{ r } }_{ L^2 ( Z ) }^2
                + D^2 \norm{ \hat{ \meanop } }_{ \op }^2
            \right)
        \right) \\
        &= 3 \norm{ \hat{ \Phi } }^2_{ \infty }
        \left(
            C_{ 0 }^2 + L^2 \norm{ \hat{ r } }_{ L^2 ( Z ) }^2
                + L^2 D^2 \norm{ \hat{ \meanop } }_{ \op }^2
        \right) \defeq \rho \left( \hat{ \Phi }, \hat{ r }, \hat{ \meanop } \right). \qedhere
    \end{align*}
\end{proof}
\begin{lemm}
    \label{lem: bound u_m - grad}
    In the procedure of Algorithm \ref{algo: sagdiv} we have
    \begin{equation*}
        \norm{
            \mean_{ \bz_{ m } } \left[
                \nabla \risk ( \hat{ h }_{ m-1 } ) - u_{ m }
            \right]
        }
        \leq
        \kappa \left( \hat{ \Phi } \right) \left(
            \norm{ \Phi - \hat{ \Phi } }_{ L^{ 2 } ( \nu_{ X } \otimes \nu_{ Z } ) }^2 + \norm{ r - \hat{ r } }_{ L^2 ( Z ) }^2 + \norm{ \meanop - \hat{ \meanop } }_{ \op }^2
        \right)^{ \frac{ 1 }{ 2 } }
    ,\end{equation*}
    where
    \begin{equation*}
        \kappa^2 \left( \hat{ \Phi } \right) \defeq 2 \max \left\{
            3 ( C_{ 0 }^2 + L^2 \mean [ Y^2 ] + L^2 D^2 ),
            2L^2 \norm{ \hat{ \Phi } }_{ \infty }^2,
            2L^2 D^2 \norm{ \hat{ \Phi } }_{ \infty }^2
        \right\}
    .\end{equation*}
\end{lemm}
\begin{proof}
    To ease the notation, we define
    \begin{align*}
        \Psi_{ m } ( Z ) &\defeq \partial_{ 2 } \ell ( r ( Z ), \meanop [ \hat{ h }_{ m-1 } ] ( Z ) ), \\
        \hat{ \Psi }_{ m } ( Z ) &\defeq \partial_{ 2 } \ell ( \hat{ r } ( Z ), \hat{ \meanop } [ \hat{ h }_{ m-1 } ] ( Z ) )
    .\end{align*}
    Let's expand the definition of $ \norm{ \cdot } $:
    \begin{align*}
        \norm{
            \mean_{ \bz_{ m } } \left[
                \nabla \risk ( \hat{ h }_{ m-1 } ) - u_{ m }
            \right]
        }
        &= \mean_{ X } \left[
            \mean_{ \bz_{ m } } \left[
                \nabla \risk ( \hat{ h }_{ m-1 } ) ( X ) - u_{ m } ( X )
            \right]^2
        \right]^{ \frac{ 1 }{ 2 } } \\
        &= \mean_{ X } \left[
            \left(
                \nabla \risk ( \hat{ h }_{ m-1 } ) ( X ) - \mean_{ \bz_{ m } } \left[ u_{ m } ( X ) \right]
            \right)^2
        \right]^{ \frac{ 1 }{ 2 } } \\
        &= \mean_{ X } \left[
            \left(
                \mean_{ Z } \left[
                    \Phi ( X, Z ) \Psi_{ m } ( Z )
                \right]
                - \mean_{ \bz_{ m } } \left[
                    \hat{ \Phi } ( X, \bz_{ m } ) \hat{ \Psi }_{ m } ( \bz_{ m } )
                \right]
            \right)^2
        \right]^{ \frac{ 1 }{ 2 } } \\
        &= \mean_{ X } \left[
            \left(
                \mean_{ Z } \left[
                    \Phi ( X, Z ) \Psi_{ m } ( Z )
                    - \hat{ \Phi } ( X, Z ) \hat{ \Psi }_{ m } ( Z )
                \right]
            \right)^2
        \right]^{ \frac{ 1 }{ 2 } }
    ,\end{align*}
    Now we add and subtract $ \hat{ \Phi } ( X, Z ) \Psi_{ m } ( Z ) $, so that
    \begin{align*}
        &\mean_{ X } \left[
            \left(
                \mean_{ Z } \left[
                    \Phi ( X, Z ) \Psi_{ m } ( Z )
                    - \hat{ \Phi } ( X, Z ) \hat{ \Psi }_{ m } ( Z )
                \right]
            \right)^2
        \right]^{ \frac{ 1 }{ 2 } } \\
        &\hspace{1cm}
        = \mean_{ X } \left[
            \left(
                \mean_{ Z } \left[
                    \Psi_{ m } ( Z ) \left(
                        \Phi ( X, Z ) - \hat{ \Phi } ( X, Z )
                    \right)
                    + \hat{ \Phi } ( X, Z ) \left(
                        \Psi_{ m } ( Z ) - \hat{ \Psi }_{ m } ( Z )
                    \right)
                \right]
            \right)^2
        \right]^{ \frac{ 1 }{ 2 } } \\
        &\hspace{1cm}
        \leq \mean_{ X } \left[
            \left(
                \norm{ \Psi_{ m } }_{ L^{ 2 } ( Z ) } \norm{ \Phi ( X, \cdot ) - \hat{ \Phi } ( X, \cdot ) }_{ L^{ 2 } ( Z ) }
                + \norm{ \hat{ \Phi } ( X, \cdot ) }_{ L^{ 2 } ( Z ) } \norm{ \Psi_{ m } - \hat{ \Psi }_{ m } }_{ L^{ 2 } ( Z ) }
            \right)^2
        \right]^{ \frac{ 1 }{ 2 } } \\
        &\hspace{1cm}
        \leq \sqrt{ 2 } \mean_{ X } \left[
            \norm{ \Psi_{ m } }_{ L^{ 2 } ( Z ) }^2 \norm{ \Phi ( X, \cdot ) - \hat{ \Phi } ( X, \cdot ) }_{ L^{ 2 } ( Z ) }^2
            + \norm{ \hat{ \Phi } ( X, \cdot ) }_{ L^{ 2 } ( Z ) }^2 \norm{ \Psi_{ m } - \hat{ \Psi }_{ m } }_{ L^{ 2 } ( Z ) }^2
        \right]^{ \frac{ 1 }{ 2 } } \\
        &\hspace{1cm}
        = \sqrt{ 2 } \left(
            \norm{ \Psi_{ m } }_{ L^{ 2 } ( Z ) }^2 \norm{ \Phi - \hat{ \Phi } }_{ L^{ 2 } ( \nu_{ X } \otimes \nu_{ Z } ) }^2
            + \norm{ \hat{ \Phi } }_{ L^{ 2 } ( \nu_{ X } \otimes \nu_{ Z } ) }^2 \norm{ \Psi_{ m } - \hat{ \Psi }_{ m } }_{ L^{ 2 } ( Z ) }^2
        \right)^{ \frac{ 1 }{ 2 } }
    ,\end{align*}
    where
    \begin{equation*}
        \norm{ \Phi }_{ L^{ 2 } ( \nu_{ X } \otimes \nu_{ Z } ) }^2 = \int_{ \mathcal{X} \times \mathcal{Z} } \Phi ( x, z )^2 p ( x ) p ( z ) \drm x \ddrm z
    \end{equation*}
    is the norm with respect to the independent coupling of the distributions of $ X $ and $ Z $.
    By Proposition \ref{prop: loss properties}.\ref{bounded growth} we have
    \begin{align*}
        \norm{ \Psi_{ m } }_{ L^{ 2 } ( Z ) }^2
        &= \mean_{ Z } \left[
            \partial_{ 2 } \ell ( r ( Z ), \meanop [ \hat{ h }_{ m-1 } ] ( Z ) )^2
        \right] \\
        &\leq \mean_{ Z } \left[
            \left(
                C_{ 0 } + L \left(
                    \abs{ r ( Z ) } + \abs{ \meanop [ \hat{ h }_{ m-1 } ] ( Z ) }
                \right)
            \right)^2
        \right] \\
        &\leq 3 \left(
            C_{ 0 }^2 + L^2 \norm{ r }_{ L^{ 2 } ( Z ) }^2 + L^2 \norm{ \meanop [ \hat{ h }_{ m-1 } ] }_{ L^{ 2 } ( Z ) }^2
        \right) \\
        &\leq 3 \left(
            C_{ 0 }^2 + L^2 \mean [ Y^2 ] + L^2 D^2
        \right)
    .\end{align*}
    It is also clear that, by Assumption \ref{assumption estimators},
    \begin{equation*}
        \norm{ \hat{ \Phi } }_{ L^2 ( \nu_{ X } \otimes \nu_{ Z } ) }^2 \leq \norm{ \hat{ \Phi } }_{ \infty }^2
    .\end{equation*}
    Finally, by Assumption \ref{assumption loss}.\ref{en: lipschitz gradients} we also have
    \begin{align*}
        \norm{ \Psi_{ m } - \hat{ \Psi }_{ m } }_{ L^{ 2 } ( Z ) }^2
        &= \mean_{ Z } \left[
            \left(
                \partial_{ 2 } \ell ( r ( Z ), \meanop [ \hat{ h }_{ m-1 } ] ( Z ) )
                - \partial_{ 2 } \ell ( \hat{ r } ( Z ), \hat{ \meanop } [ \hat{ h }_{ m-1 } ] ( Z ) )
            \right)^2
        \right] \\
        &\leq 2L^2 \left(
            \norm{ r - \hat{ r } }_{ L^{ 2 } ( Z ) }^2 + \norm{ ( \meanop - \hat{ \meanop } ) [ \hat{ h }_{ m-1 } ] }_{ L^{ 2 } ( Z ) }^2
        \right) \\
        &\leq 2L^2 \left(
            \norm{ r - \hat{ r } }_{ L^{ 2 } ( Z ) }^2 + D^2 \norm{ \meanop - \hat{ \meanop } }_{ \op }^2
        \right)
    .\end{align*}
    To combine all terms, we first define
    \begin{equation*}
        \kappa^2 \left( \hat{ \Phi } \right) \defeq 2 \max \left\{
            3 ( C_{ 0 }^2 + L^2 \mean [ Y^2 ] + L^2 D^2 ),
            2L^2 \norm{ \hat{ \Phi } }_{ \infty }^2,
            2L^2 D^2 \norm{ \hat{ \Phi } }_{ \infty }^2
        \right\}
    .\end{equation*}
    Then, it's easy to see that
    \begin{align*}
        \norm{
            \mean_{ \bz_{ m } } \left[
                \nabla \risk ( \hat{ h }_{ m-1 } ) - u_{ m }
            \right]
        } \leq \kappa \left( \hat{ \Phi } \right) \left(
            \norm{ \Phi - \hat{ \Phi } }_{ L^{ 2 } ( \nu_{ X } \otimes \nu_{ Z } ) }^2 + \norm{ r - \hat{ r } }_{ L^2 ( Z ) }^2 + \norm{ \meanop - \hat{ \meanop } }_{ \op }^2
        \right)^{ \frac{ 1 }{ 2 } }
    ,\end{align*}
    as we wanted to show.
\end{proof}

We now have everything needed to show the
\begin{proof}[Proof of Theorem \ref{thm: main theorem}]
    We start by checking that $ \risk $ is convex in $ \searchset $:
    if $ h, g \in \searchset $ and $ \lambda \in [ 0, 1 ] $, then
    \begin{align*}
        \risk ( \lambda h + ( 1 - \lambda ) g )
        &= \mean [ \loss ( r ( Z ), \meanop [ \lambda h + ( 1 - \lambda ) g ] ( Z ) ) ] \\
        &= \mean [ \loss ( r ( Z ), \lambda \meanop [ h ] ( Z ) + ( 1 - \lambda ) \meanop [ g ] ( Z ) ) ] \\
        &\leq \lambda \mean [ \loss ( r ( Z ), \meanop [ h ] ( Z ) ) ] + ( 1 - \lambda ) \mean [ \loss ( r ( Z ), \meanop [ g ] ( Z ) ) ] \\
        &= \lambda \risk ( h ) + ( 1 - \lambda ) \risk ( g )
    .\end{align*}
    By Assumption \ref{assumption on H}, there exists $ \bar{ h } \in ( \hstar + \ker \meanop ) \cap \searchset $.
    By the Algorithm \ref{algo: sagdiv} procedure, we have
    \begin{align*}
        \frac{ 1 }{ 2 } \norm{ \hat{ h }_{ m } - \hbar }^2
        &= \frac{ 1 }{ 2 } \norm{ \pi_{ \searchset } \left[ \hat{ h }_{ m-1 } - \alpha_{ m } u_{ m } \right] - \hbar }^2 \\
        &\leq \frac{ 1 }{ 2 } \norm{ \hat{ h }_{ m-1 } - \alpha_{ m } u_{ m } - \hbar }^2 \\
        &= \frac{ 1 }{ 2 } \norm{ \hat{ h }_{ m-1 } - \hbar }^2
        - \alpha_{ m } \dotprod{ u_{ m }, \hat{ h }_{ m-1 } - \hbar }
        + \frac{ \alpha_{ m }^2 }{ 2 } \norm{ u_{ m } }^2
    .\end{align*}
    After adding and subtracting $ \alpha_{ m } \dotprod{ \nabla \risk ( \hat{ h }_{ m-1 } ), \hat{ h }_{ m-1 } - \hbar } $, we are left with
    \begin{equation*}
        \frac{ 1 }{ 2 } \norm{ \hat{ h }_{ m-1 } - \hbar }^2
        - \alpha_{ m } \dotprod{ u_{ m } - \nabla \risk ( \hat{ h }_{ m-1 } ), \hat{ h }_{ m-1 } - \hbar }
        + \frac{ \alpha_{ m }^2 }{ 2 } \norm{ u_{ m } }^2
        - \alpha_{ m } \dotprod{ \nabla \risk ( \hat{ h }_{ m-1 } ), \hat{ h }_{ m-1 } - \hbar }
    .\end{equation*}
    Applying the first order convexity inequality on the last term give us, in total,
    \begin{align*}
        \frac{ 1 }{ 2 } \norm{ \hat{ h }_{ m } - \hbar }^2
        &\leq
        \frac{ 1 }{ 2 } \norm{ \hat{ h }_{ m-1 } - \hbar }^2
        - \alpha_{ m } \dotprod{ u_{ m } - \nabla \risk ( \hat{ h }_{ m-1 } ), \hat{ h }_{ m-1 } - \hbar } \\
        &\hspace{1.5cm}
        + \frac{ \alpha_{ m }^2 }{ 2 } \norm{ u_{ m } }^2
        - \alpha_{ m } ( \risk ( \hat{ h }_{ m-1 } ) - \risk ( \hbar ) )
    .\end{align*}
    Notice that, by the definition of $ \hbar $, we have $ \risk ( \hbar ) = \risk ( \hstar ) $.
    Hence, making this substitution and rearranging terms, we get
    \begin{align*}
        \risk ( \hat{ h }_{ m-1 } ) - \risk ( \hstar )
        &\leq
        \frac{ 1 }{ 2 \alpha_{ m } } \left(
            \norm{ \hat{ h }_{ m-1 } - \hbar }^2
            -
            \norm{ \hat{ h }_{ m } - \hbar }^2
        \right) \\
        &\hspace{1.5cm}+ \frac{ \alpha_{ m } }{ 2 } \norm{ u_{ m } }^2
        - \dotprod{ u_{ m } - \nabla \risk ( \hat{ h }_{ m-1 } ), \hat{ h }_{ m-1 } - \hbar }
    .\end{align*}
    Finally, summing over $ 1 \leq m \leq M $ leads to
    \begin{align}
        \label{three sums}
        \begin{split}
            \sum_{ n=1 }^{ M } \left[
                \risk ( \hat{ h }_{ m-1 } ) - \risk ( \hstar )
            \right]
            &\leq \sum_{ m=1 }^{ M } \frac{ 1 }{ 2 \alpha_{ m } } \left(
                \norm{ \hat{ h }_{ m-1 } - \hbar }^2
                -
                \norm{ \hat{ h }_{ m } - \hbar }^2
            \right) \\
            &\hspace{1cm} + \sum_{ m=1 }^{ M } \frac{ \alpha_{ m } }{ 2 } \norm{ u_{ m } }^2 \\
            &\hspace{1cm} + \sum_{ m=1 }^{ M }
            \dotprod{ \nabla \risk ( \hat{ h }_{ m-1 } ) - u_{ m }, \hat{ h }_{ m-1 } - \hbar }.
        \end{split}
    \end{align}
    The next step is to take the average of both sides with respect to $ \bz_{ 1:M } $, taking advantage of the independence between $ \bz_{ 1:M } $ and $ \dataset $, the data used to compute $ \hat{ \Phi }, \hat{ r } $ and $ \hat{ \meanop } $.
    Each summation in the RHS is then bounded separately.

    The first summation admits a deterministic bound.
    By assumption, we the diameter $ D $ of $ \searchset $ is finite.
    Hence
    \begin{align}
        \begin{split}
            \sum_{ m=1 }^{ M } \frac{ 1 }{ 2 \alpha_{ m } } \left(
                \norm{ \hat{ h }_{ m-1 } - \hbar }^2
                -
                \norm{ \hat{ h }_{ m } - \hbar }^2
            \right)
            &= \sum_{ m=2 }^{ M } \left(
                \frac{ 1 }{ 2 \alpha_{ m } } - \frac{ 1 }{ 2 \alpha_{ m-1 } } 
            \right) \norm{ \hat{ h }_{ m-1 } - \hbar }^2 \\
            &\hspace{1.5cm}+ \frac{ 1 }{ 2 \alpha_{ 1 } } \norm{ \hat{ h }_{ 0 } - \hbar }^2 - \frac{ 1 }{ 2 \alpha_{ M } } \norm{ \hat{ h }_{ M } - \hbar }^2
        \end{split} \nonumber \\
        &\leq 
        \sum_{ m=2 }^{ M } \left(
            \frac{ 1 }{ 2 \alpha_{ m } } - \frac{ 1 }{ 2 \alpha_{ m-1 } } 
        \right) D^2 + \frac{ 1 }{ 2 \alpha_{ 1 } } D^2 \nonumber \\
        &= \frac{ D^2 }{ 2 \alpha_{ M } } \label{bound first sum}
    .\end{align}
    The second summation can be bounded with the aid of Lemma \ref{lem: bound u_m}:
    \begin{equation}
        \mean_{ \bz_{ 1:M } } \left[
            \sum_{ m=1 }^{ M } \frac{ \alpha_{ m } }{ 2 } \norm{ u_{ m } }^2
        \right]
        = \frac{ \mean_{ \bz_{ 1:M } } \left[ \norm{ u_{ m } }^2 \right] }{ 2 } \sum_{ m=1 }^{ M } \alpha_{ m }
        \leq \frac{ \rho \left( \hat{ \Phi }, \hat{ r }, \hat{ \meanop } \right) }{ 2 } 
        \sum_{ m=1 }^{ M } \alpha_{ m }
        \label{bound second sum}
    .\end{equation}
    Finally, the third summation can be bounded using Lemma \ref{lem: bound u_m - grad}.
    Let $ \mean_{ \bz_{ -m } } $ denote the expectation with respect to $ \bz_{ 1 }, \dots, \bz_{ m-1 }, \bz_{ m+1 }, \dots, \bz_{ M } $ and notice that
    \begin{align*}
        \mean_{ \bz_{ 1:M } } \left[
            \dotprod{ \nabla \risk ( \hat{ h }_{ m-1 } ) - u_{ m }, \hat{ h }_{ m-1 } - \hbar }
        \right]
        &= \mean_{ \bz_{ -m } } \left[
            \mean_{ \bz_{ m } } \left[
                \dotprod{ \nabla \risk ( \hat{ h }_{ m-1 } ) - u_{ m }, \hat{ h }_{ m-1 } - \hbar }
            \right]
        \right] \\
        &= \mean_{ \bz_{ -m } } \left[
                \dotprod{
                    \mean_{ \bz_{ m } } \left[
                        \nabla \risk ( \hat{ h }_{ m-1 } ) - u_{ m }
                    \right],
                    \hat{ h }_{ m-1 } - \hbar
                }
            \right] \\
        &= \mean_{ \bz_{ -m } } \left[
                \norm{
                    \mean_{ \bz_{ m } } \left[
                        \nabla \risk ( \hat{ h }_{ m-1 } ) - u_{ m }
                    \right]
                }
                \norm{
                    \hat{ h }_{ m-1 } - \hbar
                }
            \right] \\
        &\leq D \mean_{ \bz_{ -m } } \left[
                \norm{
                    \mean_{ \bz_{ m } } \left[
                        \nabla \risk ( \hat{ h }_{ m-1 } ) - u_{ m }
                    \right]
                }
            \right]
    .\end{align*}
    Then, applying Lemma \ref{lem: bound u_m - grad} and setting $ \tau \defeq D \kappa $ we get
    \begin{align}
        \begin{split}
            &\mean_{ \bz_{ 1:M } } \left[
                \dotprod{ \nabla \risk ( \hat{ h }_{ m-1 } ) - u_{ m }, \hat{ h }_{ m-1 } - \hbar }
            \right] \\
            &\hspace{2cm}
            \leq \tau \left( \hat{ \Phi } \right) \left(
                \norm{ \Phi - \hat{ \Phi } }_{ L^{ 2 } ( \nu_{ X } \otimes \nu_{ Z } ) }^2 + \norm{ r - \hat{ r } }_{ L^2 ( Z ) }^2 + \norm{ \meanop - \hat{ \meanop } }_{ \op }^2
            \right)^{ \frac{ 1 }{ 2 } }.
        \end{split}
        \label{bound third sum}
    \end{align}
    All that is left to do is to apply equations (\ref{three sums}), (\ref{bound first sum}), (\ref{bound second sum}) and (\ref{bound third sum}) along with the inequality which defines convexity.
    Let $ \hat{ h } \defeq \frac{ 1 }{ M } \sum_{ m=1 }^{ M } \hat{ h }_{ m-1 } $ and $ \xi \defeq \rho / 2 $.
    Then:
    \begin{align*}
        &\mean_{ \bz_{ 1:M } } \left[
            \risk ( \hat{ h } ) - \risk ( \hstar )
        \right] \\
        &\hspace{1cm}
        \leq \frac{ 1 }{ M } \sum_{ m=1 }^{ M } \mean_{ \bz_{ 1:M } } \left[
            \risk ( \hat{ h }_{ m } ) - \risk ( \hstar )
        \right] \\
        &\hspace{1cm}
        \leq
        \frac{ D^2 }{ 2 M \alpha_{ M } }
        + \xi \left( \hat{ \Phi }, \hat{ r }, \hat{ \meanop } \right) \frac{ 1 }{ M } \sum_{ m=1 }^{ M } \alpha_{ m } \\
        &\hspace{2.5cm}
        + \tau \left( \hat{ \Phi } \right) \left(
            \norm{ \Phi - \hat{ \Phi } }_{ L^{ 2 } ( \nu_{ X } \otimes \nu_{ Z } ) }^2 + \norm{ r - \hat{ r } }_{ L^2 ( Z ) }^2 + \norm{ \meanop - \hat{ \meanop } }_{ \op }^2
        \right)^{ \frac{ 1 }{ 2 } }
    .\qedhere\end{align*} 
\end{proof}


\end{apendicesenv}

% ----------------------------------------------------------
% Anexos
% ----------------------------------------------------------

% \begin{anexosenv}

% \partanexos

% \end{anexosenv}

%---------------------------------------------------------------------
% ÍNDICE REMISSIVO
%---------------------------------------------------------------------
\phantompart
\printindex

\end{document}
