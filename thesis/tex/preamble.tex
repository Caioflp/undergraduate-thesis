%------------------------------------------------
% Packages
%------------------------------------------------

% \usepackage{microtype} % para melhorias de justificação
\usepackage{indentfirst}
\usepackage{csquotes}


% Gráficos 
\usepackage{color}
\usepackage{caption}
\usepackage{subcaption}
\usepackage{multirow}
\usepackage{graphicx}
\usepackage{pdfpages}
\graphicspath{{../images/}}

% Matemáticos 
\usepackage{bbm}
\usepackage{amsthm, amssymb, amsmath, mathtools}

% Outros 
\setlength{\marginparwidth}{2cm} % This will be overruled later, but is needed to that todonotes does't issue a warning
\usepackage[
    colorinlistoftodos,
    prependcaption,
    textsize=tiny,
    textwidth=15mm,
]{todonotes}
\usepackage{xargs}
\usepackage{multirow}
\usepackage{listings}
\usepackage{lstbayes}
\usepackage{silence} % Silence annoying warnings

% Citações 
%\usepackage[brazilian,hyperpageref]{backref}
%\usepackage[alf]{abntex2cite}	% Citações padrão ABNT
\usepackage[style=abnt]{biblatex}
\addbibresource{../bib/bibliography.bib}  
\WarningsOff[biblatex]

% My local macros
\usepackage{local-macros}

% \renewcommand{\backrefpagesname}{Citado na(s) página(s):~}
% % Texto padrão antes do número das páginas
% \renewcommand{\backref}{}
% % Define os textos da citação
% \renewcommand*{\backrefalt}[4]{
% 	\ifcase #1 %
% 		Nenhuma citação no texto.%
% 	\or
% 		Citado na página #2.%
% 	\else
% 		Citado #1 vezes nas páginas #2.%
% 	\fi}%
% ---

% % Changes title font to latex default
% \renewcommand{\ABNTEXchapterfont}{\fontfamily{ptm}\fontseries{b}\selectfont}

%----------------------------------------
%------- Capa e Folha de Rosto ----------
%----------------------------------------

\newcommand\subtitulo[1]{\def\@subtitulo{#1}}
\newcommand{\imprimirsubtitulo}{\@subtitulo}

\renewcommand{\imprimircapa}{%
	\begin{capa}%
	\center
		\ABNTEXchapterfont\Large \MakeUppercase{\imprimirinstituicao}
		\\\vspace*{4cm}
		{\ABNTEXchapterfont\large \MakeUppercase{\imprimirautor}}
		\vfill
		\begin{center}
		\ABNTEXchapterfont\bfseries\large\MakeUppercase{\imprimirtitulo}%\normalfont\MakeUppercase{: \imprimirsubtitulo}
		\end{center}
		\vfill
		\normalfont\large\imprimirlocal
		\\\normalfont\large\imprimirdata
		\vspace*{1cm}
	\end{capa}
}

\makeatletter
\renewcommand{\folhaderostocontent}{
  \begin{center}

    %\vspace*{1cm}
    {\ABNTEXchapterfont\large\MakeUppercase{\imprimirautor}}
	
    \vspace*{\fill}\vspace*{\fill}
    \begin{center}
      \ABNTEXchapterfont\bfseries\large\MakeUppercase{\imprimirtitulo}%\normalfont\MakeUppercase{: \imprimirsubtitulo}
    \end{center}
    \vspace*{\fill}
	
    \abntex@ifnotempty{\imprimirpreambulo}{%
      \hspace{7.5cm}
      \begin{minipage}{.5\textwidth}
      	\SingleSpacing
         \imprimirpreambulo
         \\\\
         Advisor: \imprimirorientador
       \end{minipage}%
       \vspace*{\fill}
    }%

    % {\large\imprimirorientadorRotulo~\imprimirorientador\par}
    % \abntex@ifnotempty{\imprimircoorientador}{%
    %    {\large\imprimircoorientadorRotulo~\imprimircoorientador}%
    % }%
    \vspace*{\fill}

    {\large\imprimirlocal}
    \par
    {\large\imprimirdata}
    \vspace*{1cm}

  \end{center}
}
\makeatother


\titulo{Nonparametric Instrumental Variable Regression Through Kernel Methods and Stochastic Gradients}
\autor{Caio F. Lins Peixoto}
\local{Rio de Janeiro}
\data{2023}
\instituicao{%
  Fundação Getulio Vargas \\
  \par
  School of Applied Mathematics
}
\tipotrabalho{Bachelor's Dissertation (Undergraduate)}

\preambulo{Bachelor's dissertation presented to the School of Applied
Mathematics (FGV/EMAp) to obtain the Bachelor's degree in Applied Mathematics.
\\ \\ Area of Study: Nonparametric Regression, Instrumental Variables, Kernel Methods, Stochastic Optimization, Machine Learning.}

\orientador{Yuri F. Saporito}

% Se o seu texto tem subtítulo. 
% Se não tiver, altere o arquivo capa_folha_rosto_tex
% \subtitulo{This is my subtitle}

%---------------------------------------------
%-------------------- PDF --------------------
%---------------------------------------------

% alterando o aspecto da cor azul
\definecolor{blue}{RGB}{41,5,195}

% informações do PDF
\makeatletter
\hypersetup{
     	%pagebackref=true,
		pdftitle={\@title}, 
		pdfauthor={\@author},
    	pdfsubject={\imprimirpreambulo},
	    pdfcreator={LaTeX with abnTeX2},
		pdfkeywords={abnt}{latex}{abntex}{abntex2}{trabalho acadêmico}, 
		colorlinks=true,       		% false: boxed links; true: colored links
    	linkcolor=blue,          	% color of internal links
    	citecolor=blue,        		% color of links to bibliography
    	filecolor=magenta,      		% color of file links
		urlcolor=blue,
		bookmarksdepth=4
}
\makeatother

% Posiciona figuras e tabelas no topo da página quando adicionadas sozinhas
% em um página em branco. Ver https://github.com/abntex/abntex2/issues/170
\makeatletter
\setlength{\@fptop}{5pt} % Set distance from top of page to first float
\makeatother

%---------------------------------------
%--------- Mais configurações-----------
%---------------------------------------

% Possibilita criação de Quadros e Lista de quadros.
% Ver https://github.com/abntex/abntex2/issues/176
\newcommand{\quadroname}{Quadro}
\newcommand{\listofquadrosname}{Lista de quadros}

\newfloat[chapter]{quadro}{loq}{\quadroname}
\newlistof{listofquadros}{loq}{\listofquadrosname}
\newlistentry{quadro}{loq}{0}

% configurações para atender às regras da ABNT
\setfloatadjustment{quadro}{\centering}
\counterwithout{quadro}{chapter}
\renewcommand{\cftquadroname}{\quadroname\space} 
\renewcommand*{\cftquadroaftersnum}{\hfill--\hfill}

\setfloatlocations{quadro}{hbtp} % Ver https://github.com/abntex/abntex2/issues/176

%-----------------------------------------------------
%--------------------- Margens -----------------------
%-----------------------------------------------------

\setlrmarginsandblock{3cm}{2cm}{*} % The correct is 3/2
\setulmarginsandblock{3cm}{2cm}{*}
\checkandfixthelayout

%-----------------------------------------------------
%------ Espaçamentos entre linhas e parágrafos -------
%-----------------------------------------------------

% O tamanho do parágrafo é dado por:
\setlength{\parindent}{1.3cm}

% Controle do espaçamento entre um parágrafo e outro:
\setlength{\parskip}{0.2cm}  % tente também \onelineskip

% compila o índice
\makeindex

%------------------------------------------------------
% Personal Macros
%------------------------------------------------------

\renewcommand{\quadroname}{Chart}


% Removes paragraph indenting in description environments
\setlist[description]{leftmargin=0pt, font={\bfseries\sffamily}}
% Sets enumeration type
\setlist[enumerate]{label=(\roman*)}


%------------------------------------------------------
% Todo notes
%------------------------------------------------------

\WarningsOff[todonotes]

\newcommandx{\unsure}[2][1=]{\todo[linecolor=red,backgroundcolor=red!25,bordercolor=red,#1]{#2}}
\newcommandx{\change}[2][1=]{\todo[linecolor=blue,backgroundcolor=blue!25,bordercolor=blue,#1]{#2}}
\newcommandx{\info}[2][1=]{\todo[linecolor=OliveGreen,backgroundcolor=OliveGreen!25,bordercolor=OliveGreen,#1]{#2}}
\newcommandx{\improvement}[2][1=]{\todo[linecolor=violet,backgroundcolor=violet!25,bordercolor=violet,#1]{#2}}
\newcommandx{\thiswillnotshow}[2][1=]{\todo[disable,#1]{#2}}

% ------------------------------------------------------------ 
% Theorem-like environments
% ------------------------------------------------------------ 
\newtheoremstyle{fancy} % name
        {0.7em}                    % Space above
        {0.5em}                    % Space below
        {\normalfont}                   % Body font
        {}                           % Indent amount
        {\bfseries\sffamily}                   % Theorem head font
        {}                          % Punctuation after theorem head
        {0.5em}                       % Space after theorem head
        {}  % Theorem head spec (can be left empty, meaning ‘normal’)
\newtheoremstyle{regular} % name
        {0.7em}                    % Space above
        {0.5em}                    % Space below
        {}                   % Body font
        {}                           % Indent amount
        {\bfseries\sffamily}                   % Theorem head font
        {}                          % Punctuation after theorem head
        {0.5em}                       % Space after theorem head
        {}  % Theorem head spec (can be left empty, meaning ‘normal’)
\swapnumbers

\theoremstyle{fancy} % default
% \newtheorem{name}[shared counter]{text}[upper counter]
\newtheorem{thm}{Theorem}[chapter]
\newtheorem*{thm*}{Theorem}
\newtheorem{lemm}[thm]{Lemma}
\newtheorem*{lemm*}{Lemma}
\newtheorem{prop}[thm]{Proposition}
\newtheorem*{prop*}{Proposition}
\newtheorem{cor}[thm]{Corollary}
\newtheorem{corr}[thm]{Corollary}

\theoremstyle{regular}
\newtheorem{deff}[thm]{Definition}
\newtheorem{rem}[thm]{Remark}
\newtheorem*{axiom}{Axiom}
