\setlength{\absparsep}{18pt} 

\begin{resumo}[Abstract]
 \begin{otherlanguage*}{english}
     Instrumental variables (IVs) are a widely used tool in econometrics to address regression problems in which some covariates are correlated with the error term.
     In the nonparametric framework, one would like to find the true functional parameter from an infinite dimensional search space.
     This work provides an introduction to IV regression, both in its parametric and nonparametric forms, and presents a novel method to tackle the latter, based on a functional variant of stochastic first order methods.
     Theoretical as well as empirical results are presented.
 \end{otherlanguage*}

 Keywords: Nonparametric Regression, Instrumental Variables, Stochastic Optimization.
\end{resumo}

\begin{resumo}[Resumo]
    Variáveis instrumentais (VIs) são uma ferramenta amplamente utilizada em econometria para tratar problemas de regressão em que algumas covariáveis são correlacionadas com termo de erro.
    No âmbito não paramétrico, o objetivo é encontrar o verdadeiro parâmetro funcional em um espaço de procura de dimensão infinita.
    Este trabalho fornece uma introdução a regressão por VI, tanto no formato paramétrico quanto não paramétrico, e apresenta um novo método para tratar o último, baseado em uma variante funcional de métodos estocásticos de primeira ordem.
    Resultados teóricos e empíricos são apresentados.

    Palavras-chave: Regressão Não Paramética, Variáveis Instrumentais, Otimização Estocástica.
\end{resumo}
