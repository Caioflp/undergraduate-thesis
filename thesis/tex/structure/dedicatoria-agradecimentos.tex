\newpage

\begin{dedicatoria}
    \vspace*{\fill}
    %\noindent
    \hfill
    \begin{minipage}{.6\textwidth}
        Para Helder e Mariza.
    \end{minipage}
\end{dedicatoria}
 
\begin{agradecimentos}
    In consideration for those mentioned here, this section will be written in my mother tongue, Portuguese.

    Este trabalho marca o final da minha graduação na EMAp e, portanto, gostaria de agradecer a todos aqueles que, de alguma forma, contribuíram para essa jornada.

    Agradeço, em primeiro lugar, a meus pais, Helder e Mariza, pelo apoio incondicional que sempre deram a mim.
    Sem o amor, a dedicação e o incentivo de vocês, eu jamais estaria onde estou hoje.
    
    Agradeço à minha namorada, Fran, por sempre estar ao meu lado e acreditar em mim, mesmo quando a distância parecia ser grande demais.
    Saber que você estaria me esperando em BH me dava forças para continuar seguindo em frente.

    Agradeço aos meus amigos da Bald Comics, Adame e Tulio, por serem minha companhia constante e por tornarem mais leves todos os perrengues da graduação.
    Sou muito sortudo de ter amigos tão talentosos e atenciosos quanto vocês.
    Aos meus amigos de mais longa data, também deixo meus mais sinceros agradecimentos: Gabriel, Elias, Tiago, Miguel e Motta.
    Mesmo distantes fisicamente, vocês se fizeram presentes em diversos momentos, e isso teve um valor inestimável.

    Agradeço ao Centro de Desenvolvimento da Matemática e Ciências (CDMC) da FGV, por ter me proporcionado a oportunidade de cursar minha graduação na EMAp, algo que mudou completamente o rumo da minha vida.
    Em particular, agradeço a Cássia Pessanha e Luziel Claret pelo suporte constante, e ao diretor do CDMC, César Camacho, por ter idealizado esse projeto e por atuar constantemente para mantê-lo viável.

    Agradeço a meu colaborador Yuri Resende, por todas as discussões proveitosas a respeito deste projeto.

    Por último, mas de longe não menos importante, agradeço ao meu orientador, Yuri Saporito.
    Foram, ao total, três anos em que tive o privilégio de poder tê-lo como professor, colaborador e amigo.
    Você me acompanhou desde quando eu não sabia o que era uma variável aleatória até agora, nunca deixou de acreditar no meu potencial e sempre me incentivou a buscar o que é melhor para mim.
    Por isso, sempre serei grato.
\end{agradecimentos}

\begin{epigrafe}
\vspace*{\fill}

\begin{flushright}
    \hspace{7.5cm}
    \textit{
        ``
        It was as though applied Mathematics was my spouse, and pure Mathematics was my secret lover.
        ''
    } \\
        \textit{Edward Frenkel}
\end{flushright}
\end{epigrafe}
