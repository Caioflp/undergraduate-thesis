\chapter{Introduction}

When choosing a method to tackle a regression problem $ Y = h_{ \theta } ( X ) + \varepsilon $, one is often concerned with certain properties of the resulting estimator $ \hat{ \theta } $, such as consistency and unbiasedness.
The widely employed linear model estimator, for example, enjoys these properties if the problem at hand satisfies its usual hypotheses, which include assuming that the noise term $ \varepsilon $ is uncorrelated with the covariate $ X $.
However, in practical scenarios it is very common to have external unmeasured factors influence both $ X $ and $ \varepsilon $, which introduces some correlation between these variables.
In these cases, where $ X $ is said to be \emph{endogenous}, methods based on ordinary least squares turn out to be biased and inconsistent.
To address this problem, econometric practitioners invented the method of instrumental variables, which uses a third measurable quantity, $ Z $, that influences $ X $ but is uncorrelated with $ \varepsilon $, to correct the faults in ordinary least squares estimators.

This work deals with the nonparametric version of this problem, where $ h_{ \theta } $ is a member of an infinite dimensional space.
In recent years we have seen a ressurgence of interest in nonparametric instrumental variable (NPIV) regression, with multiple methods being devised and published, such as the ones in \cite{singh2019}, \cite{deepiv2017}, \cite{deepgmm2019} and \cite{dualiv2020}.

About the structure of this document, in Chapter 2 we provide an overview of instrumental variables in econometric practice and present the Two Stages Least Squares estimator, which generalizes the linear model to the case where some covariates are endogenous.
In Chapter 3 we show how NPIV regression can be seen as an ill-posed linear inverse problem and present two well established methods to tackle it.
Chapter 4 then derives and analyses SAGD-IV, our own estimator for NPIV regression problems based on a functional variant of stochastic first order methods.
A convergence speed bound is shown, as well as an empirical performance check using artificial data.
Our method stands out because it can increase the quality of the estimator only using additional samples from $ Z $, and our theoretical results, albeit not as strong, are proved under weak assumptions when compared to the literature.
