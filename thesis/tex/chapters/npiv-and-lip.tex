\chapter{Nonparametric Instrumental Variable \\Regression and Linear Inverse Problems}

Having defined what we mean by nonparametric regression instrumental variable (NPIV) regression, we now present two well established approaches, namely the ones in \cite{newey2003} and \cite{darolles2011}, the latter of which will serve as a starting point for our method.
% We will describe the idea behind Newey's seminal paper on NPIV \cite{newey2003}, which develops a nonparametric analog of 2SLS, and them move on to the more inverse problem (IP) oriented approach presented in \cite{florens2007} as well as \cite{darolles2011}.

\section{NPIV as an ill-posed linear inverse problem}

Recalling the notation presented in section \ref{sec: nonparametric}, we want to find $ \hstar \in L^{ 2 } ( X ) $ which satisfies
\begin{equation}
    \label{eq: basic equation}
    Y = \hstar ( X ) + \varepsilon
,\end{equation}
where $ \mean [ \varepsilon \mid Z ] = 0 $.
Letting $ r ( Z ) \defeq \mean [ Y \mid Z ] $ and assuming that $ Y $ has a finite second moment, we have $ r \in L^{ 2 } ( Z ) $, so that (\ref{eq: basic equation}) is equivalent to
\begin{equation}
    \label{eq: linear equation}
    r = \meanop [ \hstar ]
,\end{equation}
where $ \meanop : L^{ 2 } ( X ) \to L^{ 2 } ( Z ) $ is the conditional expectation operator.
Assume that the joint distribution of $ ( X, Z ) $ is absolutely continuous with respect to Lebesgue measure in $ \R^{ d_{ X } + d_{ Z } } $, so that we may rewrite (\ref{eq: linear equation}) as
\begin{equation}
    \label{}
    r ( z ) = \int_{ \R^{ d_{ X } } } \hstar ( x ) p_{ X \mid Z } ( x \mid z ) \drm x
,\end{equation}
where $ p_{ X\mid Z } ( x \mid z ) $ is the conditional density of $ X $ given $ Z $.
This is a Fredholm integral equation of the first kind \cite{kress89} which, due to the nature of our problem, will most likely be ill-posed.
We dedicate some space to make this statement precise.

% {\color{red}Talk about the three ways in which the problem can be ill posed, relate them with identification, talk about compactness of $ \meanop $ and it being Hilbert-Schmidt, say what we are going to assume in what follows.}

Equation (\ref{eq: linear equation}) would describe a well-posed problem if the operator $ \meanop $ were invertible and the inverse $ \meanop^{ -1 } $, continuous.
There are three ways in which these conditions may be violated.
One of them was already discussed in section \ref{sec: identification}, the identification problem.
It corresponds to non-injectivity of $ \meanop $, that is, to $ \ker \meanop \neq \left\{ 0 \right\} $.
A possible correction for this problem is to look for the least norm solution.

Another violation happens if $ \meanop $ is not surjective, which leaves the possibility of $ r \notin \image ( \meanop ) $.
A way to bypass this difficulty is to project $ r $ onto the subspace $ \image ( \meanop ) $ of $ L^{ 2 } ( Z ) $ and find the inverse image of this projection.
However, the orthogonal projection onto a subspace is only well defined for $ \closure{\image ( \meanop )} $, since we need the subspace to be closed, which may not be the case for $ \image ( \meanop ) $.
Hence, we would need to assume that the projection of $ r $ onto $ \closure{\image ( \meanop )} $ belongs to $ \image ( \meanop ) $.
The methods we will analyze assume that the model (\ref{eq: basic equation}) is correctly specified and, therefore, that $ r \in \image ( \meanop ) $, so this won't be a concern for us.
The interested reader may consult \cite{florens2007} for ways to proceed without this assumption.

The last way for a problem such as (\ref{eq: basic equation}) to be ill-posed is to have $ \meanop $ injective and $ r \in \image ( \meanop ) $, but $ \meanop^{ -1 } : \image ( \meanop ) \to L^{ 2 } ( X ) $ discontinuous.
By the Open Mapping Theorem, if $ \image ( \meanop ) $ is closed, then $ \meanop : L^{ 2 } ( X ) \to \image ( \meanop ) $ is an open map and the inverse is automatically continuous.
Hence, we must require that $ \image ( \meanop ) $ is not closed.
A prototypical example for this situation is when $ \meanop $ is a compact operator with infinite dimensional range.
If $ \image ( \meanop ) $ were closed, since $ \id_{ \image ( \meanop ) } = \meanop^{ -1 } \circ \meanop $ it would be compact, which would imply compactness of the closed unit ball of $ \image ( \meanop ) $ and, hence, $ \dim \image ( \meanop ) < \infty $ \cite{florens2007}.

Aside from requiring $ r \in \image ( \meanop ) $, we leave the possibility of any other form of ill-posedness to be present in our problem until further notice.

\section{Nonparametric 2SLS}

In \cite{newey2003}, the author's
