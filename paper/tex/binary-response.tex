\section{Binary response models}
We want to be able to employ the same risk minimization procedure:
\begin{equation}
    \label{risk minimization}
    \argmin_{ h \in \mathcal{F} } \risk ( h )
    = \argmin_{ h \in \mathcal{F} } \mean_{ Z } \left[
        \ell ( r_{ 0 } ( Z ), \meanop [ h ] ( Z ) )
    \right]
.\end{equation}
Let's see what data generating procedure makes this possible.
Firstly, let
\begin{equation}
    \label{Y explicit bernoulli}
    Y \mid X, \varepsilon \sim \ber ( \sigma ( \hstar ( X ) + \varepsilon ) )
,\end{equation}
where $ \sigma $ is the logistic function, $ \mean [ \varepsilon \mid X ] \neq 0 $ and $ \mean [ \varepsilon \mid Z ] = 0 $.
For (\ref{risk minimization}) to make sense, we'd like $ r_{ 0 } ( Z ) = \mean [ Y \mid Z ] $ and $ \meanop [ \hstar ] ( Z ) = \mean [ \hstar ( X ) \mid Z ] $ to be close according to a suitable loss function $ \ell $, at least close enough so that $ \hstar $ is a solution to (\ref{risk minimization}).
Let's see if this is the case under (\ref{Y explicit bernoulli}):
\begin{equation*}
    \mean [ Y \mid Z ] = \prob [ Y = 1 \mid Z ]
,\end{equation*}

Assuming (\ref{Y explicit bernoulli}), we may compute this conditioning on $ X $ and $ \varepsilon $ and then integrating them out:
\begin{align*}
    \prob [ Y = 1 \mid Z = z ]
    &= \int_{ \mathcal{X} \times \R } \prob [ Y = 1 \mid Z = z, X = x, \varepsilon = e ] p_{ X, \varepsilon \mid Z } ( x, e \mid z ) \drm x \ddrm \varepsilon \\
    &= \int_{ \mathcal{X} \times \R } \sigma ( \hstar ( x ) + e ) p_{ X, \varepsilon \mid Z } ( x, e \mid z ) \drm x \ddrm \varepsilon \\
    &= \mean [ \sigma ( \hstar ( X ) + \varepsilon ) \mid Z = z ]
.\end{align*}
There are a two main problems here.
The first one is that $ \varepsilon $ appears inside $ \sigma $ and, hence, does not vanish after conditioning on $ Z = z $.
I cannot think of a way to remove it without assuming known the distribution of $ \varepsilon $ given $ X $, which is prohibitive.
The second problem is that, even if there was no $ \varepsilon $, the expectation is outside the function $ \sigma $.
In order for (\ref{risk minimization}) to work under (\ref{Y explicit bernoulli}), we'd like set
\begin{equation*}
    \ell ( y, y' ) = \BCE ( y, \sigma ( y' ) )
,\end{equation*}
where $ \BCE $ is the binary cross entropy loss function:
\begin{equation*}
    \BCE ( y, p ) = - [ y \log p + ( 1 - y ) \log ( 1 - p ) ]
.\end{equation*}
That is, we'd like to have $ \sigma \left( \mean [ h ( X ) \mid Z ] \right) $ inside $ \risk ( h ) $, instead of $ \mean \left[ \sigma \left( h ( X ) \right) \mid Z \right] $.

The second option is to set
\begin{equation}
    \label{Y is indicator}
    Y = \ind [ \hstar ( X ) + \varepsilon > 0 ]
.\end{equation}
Here, we have
\begin{align}
    \mean [ Y \mid Z = z ]
    &= \prob [ \hstar ( X ) + \varepsilon > 0 \mid Z = z ]
.\end{align}
To try to make this lead somewhere, let's define $ \eta = \hstar ( X ) - \mean [ \hstar ( X ) \mid Z ] + \varepsilon $, so that
\begin{equation*}
    Y = \ind [ \mean [ \hstar ( X ) \mid Z ] + \eta > 0]
\end{equation*}
and $ \mean [ \eta \mid Z ] = 0 $.
Let $ t ( Z ) = \mean [ \hstar ( X ) \mid Z ] $.
This implies
\begin{align*}
    \mean [ Y \mid Z ]
    &= \prob [ t ( Z ) + \eta > 0 \mid Z ] \\
    &= 1 - F_{ \eta \mid Z } ( - t ( Z ) )
.\end{align*}
Hence, we have
\begin{equation*}
    t ( Z ) = - F_{ \eta \mid Z }^{ -1 } ( r_{ 0 } ( Z ) - 1 )
.\end{equation*}
Or, equivalently:
\begin{equation*}
    \mean [ \hstar ( X ) \mid Z ] = - F^{ -1 }_{ \eta \mid Z } \left( \mean [ Y \mid Z ] - 1 \right)
.\end{equation*}
This looks promising:
If we assume to know the conditional distribution of $ \eta $ given $ Z $, we have a couple of options.
We can minimize
\begin{equation*}
    \BCE ( r_{ 0 } ( Z ) , 1 - F_{ \eta \mid Z } ( - \mean [ h ( X ) \mid Z ] ) )
,\end{equation*}
or
\begin{equation*}
     \left( \mean [ h ( X ) \mid Z ] + F_{ \eta \mid Z }^{ -1 } ( r_{ 0 } ( Z ) - 1 ) \right)^2
.\end{equation*}
This assumption was used on the paper ``Nonparametric Instrumental Variable Estimation of Binary Response Models'', by P. L. Florens, from where I took the ideas for these calculations.

In an unpublished version of that paper, they assume that $ \eta = \frac{ 1 }{ \zeta ( Z ) } v $, where $ v \mid Z \sim \operatorname{KnownDistribution}(0, \sigma_{ v }^2) $.
This implies
\begin{align*}
    \mean [ Y \mid Z ]
    &= \prob [ t ( Z ) + \eta > 0 \mid Z ] \\
    &= \prob [ t ( Z ) + \frac{ v }{ \zeta ( Z ) } > 0 \mid Z ] \\
    &= \prob [ v > - t ( Z ) \zeta ( Z ) \mid Z ] \\
    &= 1 - F_{ v } ( - t ( Z ) \zeta ( Z ) ) \\
    &\defeq 1 - F_{ v } ( - \gamma ( Z ) )
.\end{align*}
Equivalently, this means that
\begin{equation*}
    \gamma ( Z ) =  - F_{ v }^{ -1 } ( 1 - \mean [ Y \mid Z ] )
,\end{equation*}
where $ \gamma ( Z ) = \mean [ \hstar ( X ) \mid Z ] \zeta ( Z ) $.
They proceed to use $ \gamma $ to estimate $ r_{ 0 } $ (this involves splitting $ Z $ into two parts and is the main contribution in their article) and then use this estimate of $ r_{ 0 } $ to estimate $ \hstar $ through Tikhonov regularization.

However, on the published version, the authors assume that $ \eta $ is \emph{independent} of $ Z $, which is good for us.
