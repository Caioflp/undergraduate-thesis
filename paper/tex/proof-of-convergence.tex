\section{Proof of convergence}

The first problem is proving our sequence of estimates is, in fact, contained in $ L^{ 2 } ( X ) $.
This amounts to proving $ u_{ m } \in L^{ 2 } ( X ) $ for every $ m $.
It's not even immediate why $ u_{ h } ( x ) = \Phi ( x, Z ) \xi_{ h } ( Z ) $ (the unbiased gradient when we know $ r_{ 0 }, \Phi $ and $ \mathcal{T} $) belongs to $ L^{ 2 } ( X ) $\improvement{ We'll need to bound the norm of $ u_{ m } $ by a constant later in the proof. }

After doing this, we check that $ \risk $ is convex in $ \mathcal{F} $:
if $ h, g \in \mathcal{F} $ and $ \lambda \in [ 0, 1 ] $, then
\begin{align*}
    \risk ( \lambda h + ( 1 - \lambda ) g )
    &= \mean [ \loss ( r_{ 0 } ( Z ), \mathcal{T} [ \lambda h + ( 1 - \lambda ) g ] ( Z ) ) ] \\
    &= \mean [ \loss ( r_{ 0 } ( Z ), \lambda \mathcal{T} [ h ] ( Z ) + ( 1 - \lambda ) \mathcal{T} [ g ] ( Z ) ) ] \\
    &\leq \lambda \mean [ \loss ( r_{ 0 } ( Z ), \mathcal{T} [ h ] ( Z ) ) ] + ( 1 - \lambda ) \mean [ \loss ( r_{ 0 } ( Z ), \mathcal{T} [ g ] ( Z ) ) ] \\
    &= \lambda \risk ( h ) + ( 1 - \lambda ) \risk ( g )
.\end{align*}
To lighten the notation, we denote the norm and inner product in $ L^{ 2 } ( X ) $ by $ \norm{ \cdot } $ and $ \dotprod{ \cdot, \cdot } $, respectively.
By the Algorithm \ref{algo: functional sgd} procedure, we have
\begin{align*}
    \frac{ 1 }{ 2 } \norm{ \hat{ h }_{ m } - \hstar }^2
    &= \frac{ 1 }{ 2 } \norm{ \hat{ h }_{ m-1 } - \alpha_{ m } u_{ m } - \hstar }^2 \\
    &= \frac{ 1 }{ 2 } \norm{ \hat{ h }_{ m-1 } - \hstar }^2
    - \alpha_{ m } \dotprod{ u_{ m }, \hat{ h }_{ m-1 } - \hstar }
    + \frac{ \alpha_{ m }^2 }{ 2 } \norm{ u_{ m } }^2
.\end{align*}
After adding and subtracting $ \alpha_{ m } \dotprod{ \nabla \risk ( \hat{ h }_{ m-1 } ), \hat{ h }_{ m-1 } - \hstar } $, we are left with
\begin{equation*}
    \frac{ 1 }{ 2 } \norm{ \hat{ h }_{ m-1 } - \hstar }^2
    - \alpha_{ m } \dotprod{ u_{ m } - \nabla \risk ( \hat{ h }_{ m-1 } ), \hat{ h }_{ m-1 } - \hstar }
    + \frac{ \alpha_{ m }^2 }{ 2 } \norm{ u_{ m } }^2
    - \alpha_{ m } \dotprod{ \nabla \risk ( \hat{ h }_{ m-1 } ), \hat{ h }_{ m-1 } - \hstar }
.\end{equation*}
Applying the basic convexity inequality on the last term give us, in total,
\begin{align*}
    \frac{ 1 }{ 2 } \norm{ \hat{ h }_{ m } - \hstar }^2
    &\leq
    \frac{ 1 }{ 2 } \norm{ \hat{ h }_{ m-1 } - \hstar }^2
    - \alpha_{ m } \dotprod{ u_{ m } - \nabla \risk ( \hat{ h }_{ m-1 } ), \hat{ h }_{ m-1 } - \hstar } \\
    &\hspace{1.5cm}
    + \frac{ \alpha_{ m }^2 }{ 2 } \norm{ u_{ m } }^2
    - \alpha_{ m } ( \risk ( \hat{ h }_{ m-1 } ) - \risk ( \hstar ) )
.\end{align*}
Rearranging terms, we get
\begin{align*}
    \risk ( \hat{ h }_{ m-1 } ) - \risk ( \hstar )
    &\leq
    \frac{ 1 }{ 2 \alpha_{ m } } \left(
        \norm{ \hat{ h }_{ m-1 } - \hstar }^2
        -
        \norm{ \hat{ h }_{ m } - \hstar }^2
    \right) \\
    &\hspace{1.5cm}+ \frac{ \alpha_{ m } }{ 2 } \norm{ u_{ m } }^2
    - \dotprod{ u_{ m } - \nabla \risk ( \hat{ h }_{ m-1 } ), \hat{ h }_{ m-1 } - \hstar }
.\end{align*}
Finally, summing over $ 1 \leq m \leq M $ leads to
\begin{align*}
    \sum_{ n=1 }^{ M } \left[
        \risk ( \hat{ h }_{ m-1 } ) - \risk ( \hstar )
    \right]
    &\leq \sum_{ m=1 }^{ M } \frac{ 1 }{ 2 \alpha_{ m } } \left(
        \norm{ \hat{ h }_{ m-1 } - \hstar }^2
        -
        \norm{ \hat{ h }_{ m } - \hstar }^2
    \right) \\
    &\hspace{1cm} + \sum_{ m=1 }^{ M } \frac{ \alpha_{ m } }{ 2 } \norm{ u_{ m } }^2 \\
    &\hspace{1cm} - \sum_{ m=1 }^{ M }
    \dotprod{ u_{ m } - \nabla \risk ( \hat{ h }_{ m-1 } ), \hat{ h }_{ m-1 } - \hstar }
.\end{align*}


\subsection{Option 1: averaging with respect to everything}


We then treat each term (summation) separately:
\begin{itemize}
    \item The first term is bounded using the assumption\info{Assumption} that $ \diam \mathcal{F} = D < \infty $.
    \item The bound on the second term depends on bounding $ \mean \left[ \norm{ u_{ m } }^2_{ L^{ 2 } ( X ) } \right] $ by a constant independent of $ m $ or $ M $.
    \item The third term must vanish because of the unbiasedness of $ u_{ m } $, but we don't know that our $ u_{ m } $ is unbiased, and it may very well not be.
\end{itemize}

\begin{description}[style=unboxed, leftmargin=0cm]
    \item[First term]
        By assumption, we have $ \diam \mathcal{F} = D < \infty $.
        Hence
        \begin{align*}
            \begin{split}
                \sum_{ m=1 }^{ M } \frac{ 1 }{ 2 \alpha_{ m } } \left(
                    \norm{ \hat{ h }_{ m-1 } - \hstar }^2
                    -
                    \norm{ \hat{ h }_{ m } - \hstar }^2
                \right)
                &= \sum_{ m=2 }^{ M } \left(
                    \frac{ 1 }{ 2 \alpha_{ m } } - \frac{ 1 }{ 2 \alpha_{ m-1 } } 
                \right) \norm{ \hat{ h }_{ m-1 } - \hstar }^2 \\
                &\hspace{1.5cm}+ \frac{ 1 }{ 2 \alpha_{ 1 } } \norm{ \hat{ h }_{ 0 } - \hstar }^2 - \frac{ 1 }{ 2 \alpha_{ M } } \norm{ \hat{ h }_{ M } - \hstar }^2
            \end{split} \\
            &\leq 
            \sum_{ m=2 }^{ M } \left(
                \frac{ 1 }{ 2 \alpha_{ m } } - \frac{ 1 }{ 2 \alpha_{ m-1 } } 
            \right) D^2 + \frac{ 1 }{ 2 \alpha_{ 1 } } D^2 = \frac{ D^2 }{ 2 \alpha_{ M } }
        .\end{align*}
    \item[Second term]
        Define $ \mathcal{D} $ to be the set of all observed data, that is, all of the variables in $ \data_{ \Phi }, \data_{ r_{ 0 } }, \data_{ \mathcal{T} } $ and $ \left\{ \bz_{ i } \right\}_{ i=1 }^{ M } $.
        Let's evaluate $ \mean \left[ \norm{ u_{ m } }^2 \right] $:
        \begin{equation*}
            \mean \left[
                \norm{ u_{ m } }^2
            \right]
            = \mean_{ \data } \left[
                \mean_{ X } \left[
                    \hat{ \Phi } ( X, \bz_{ m } ; \mathcal{D}_{ \Phi } )^2
                    \partial_{ 2 } \loss \left(
                        \hat{ r_{ 0 } } ( \bz_{ m } ; \mathcal{D}_{ r_{ 0 } } ),
                        \hat{ \mathcal{T} [ \hat{ h }_{ m-1 } ] } ( \bz_{ m } ; \mathcal{D}_{ \mathcal{T} } )
                    \right)^2
                \right]
            \right]
        ,\end{equation*}
        where the second expectation is with respect to a copy of $ X $ which is independent of $ \data $.
        Continuing:
        \begin{align*}
            &\mean_{ \data } \left[
                \mean_{ X } \left[
                    \hat{ \Phi } ( X, \bz_{ m } ; \mathcal{D}_{ \Phi } )^2
                    \partial_{ 2 } \loss \left(
                        \hat{ r_{ 0 } } ( \bz_{ m } ; \mathcal{D}_{ r_{ 0 } } ),
                        \hat{ \mathcal{T} [ \hat{ h }_{ m-1 } ] } ( \bz_{ m } ; \mathcal{D}_{ \mathcal{T} } )
                    \right)^2
                \right]
            \right] \\
            &\hspace{2cm}=
            \mean_{ \data } \left[
                \partial_{ 2 } \loss \left(
                    \hat{ r_{ 0 } } ( \bz_{ m } ; \mathcal{D}_{ r_{ 0 } } ),
                    \hat{ \mathcal{T} [ \hat{ h }_{ m-1 } ] } ( \bz_{ m } ; \mathcal{D}_{ \mathcal{T} } )
                \right)^2
                \mean_{ X } \left[
                    \hat{ \Phi } ( X, \bz_{ m } ; \mathcal{D}_{ \Phi } )^2
                \right]
            \right]
        .\end{align*}
        If $ \ell $ is quadratic, we have
        \begin{align*}
            &\mean_{ \data } \left[
                \partial_{ 2 } \loss \left(
                    \hat{ r_{ 0 } } ( \bz_{ m } ; \mathcal{D}_{ r_{ 0 } } ),
                    \hat{ \mathcal{T} [ \hat{ h }_{ m-1 } ] } ( \bz_{ m } ; \mathcal{D}_{ \mathcal{T} } )
                \right)^2
                \mean_{ X } \left[
                    \hat{ \Phi } ( X, \bz_{ m } ; \mathcal{D}_{ \Phi } )^2
                \right]
            \right] \\
            &\hspace{2cm}=
            \mean_{ \data } \left[
                \left(
                    \hat{ \mathcal{T} [ \hat{ h }_{ m-1 } ] } ( \bz_{ m } ; \mathcal{D}_{ \mathcal{T} } )
                    - \hat{ r_{ 0 } } ( \bz_{ m } ; \mathcal{D}_{ r_{ 0 } } )
                \right)^2
                \mean_{ X } \left[
                    \hat{ \Phi } ( X, \bz_{ m } ; \mathcal{D}_{ \Phi } )^2
                \right]
            \right]
        .\end{align*}
        \improvement{Find a way to finish this bound.
        Maybe switch the order of expectations? First in $ X $ and then in $ \data $.}

    \item[Third term]
        We have to work with
        \begin{equation*}
            \mean \left[
                \dotprod{ u_{ m } - \nabla \risk ( \hat{ h }_{ m-1 } ), \hat{ h }_{ m-1 } - \hstar }
            \right]
        .\end{equation*}
        The strategy employed on the SIP paper was to condition on the $ \sigma $-algebra generated by the training samples observed up until iteration iteration $ m - 1 $.
        In our case, that would be $ \bz_{ 1 }, \dots, \bz_{ m-1 } $.
        The problem which arises is that we no longer have measurability of $ \hat{ h }_{ m-1 } $ with respect to this $ \sigma $-algebra, as it depends on the datasets $ \data_{ \Phi }, \data_{ \mathcal{T} }, \data_{ r_{ 0 } } $, used to estimate $ \hat{ \Phi }, \hat{ \mathcal{T} } $ and $ \hat{ r }_{ 0 } $ in an offline manner.

        The other option would be to condition on more things, namely the $ \sigma $-algebra generated by $ \bz_{ 1 }, \dots, \bz_{ m - 1 }, \data_{ \Phi }, \data_{ \mathcal{T} }, \data_{ r_{ 0 } } $.
        We gain measurability of $ \hat{ h }_{ m-1 } $, but we are no longer integrating out $ \data_{ \Phi }, \data_{ \mathcal{T} }, \data_{ r_{ 0 } } $, which is needed to use some sort of unbiasedness of the estimators $ \hat{ \Phi }, \hat{ \mathcal{T} }, \hat{ r }_{ 0 } $.

        Let's try the latter and see what we end up with.
        Define $ \data_{ m-1 } \defeq \data_{ \Phi } \cup \data_{ \mathcal{T} } \cup \data_{ r_{ 0 } } \cup \left\{ \bz_{ 1 }, \dots, \bz_{ m-1 } \right\} $.
        Then,
        \begin{align*}
            \begin{split}
                &\mean_{ \data } \left[
                    \dotprod{ u_{ m } - \nabla \risk ( \hat{ h }_{ m-1 } ), \hat{ h }_{ m-1 } - \hstar }
                \right] \\
                &\hspace{2cm}
                =\mean_{ \data_{ m-1 } } \left[
                    \mean \left[
                        \dotprod{ u_{ m } - \nabla \risk ( \hat{ h }_{ m-1 } ), \hat{ h }_{ m-1 } - \hstar }
                        \mid \data_{ m-1 }
                    \right]
                \right]
            \end{split} \\
            &\hspace{2cm}
            = \mean_{ \data_{ m-1 } } \left[
                \dotprod{ 
                    \mean \left[
                        u_{ m }
                        \mid \data_{ m-1 }
                    \right]
                    - \nabla \risk ( \hat{ h }_{ m-1 } )
                    ,
                    \hat{ h }_{ m-1 } - \hstar
                }
        \right] && (\text{Justify this properly.})
        .\end{align*}
        Now we must evaluate the expression inside the inner product.
        We restrict ourselves to the quadratic case.
        Define $ \psi ( z ) = \mathcal{T} [ \hat{ h }_{ m-1 } ] ( z ) - r_{ 0 } ( z ) $ and $ \hat{ \psi } ( z ; \dataproj ) = \hat{ \mathcal{T} [ \hat{ h }_{ m-1 } ] } ( z ; \data_{ \mathcal{T} } ) - \hat{ r }_{ 0 } ( z ; \data_{ r_{ 0 } } ) $, where $ \dataproj = \data_{ \mathcal{T} } \cup \data_{ r_{ 0 } } $.
        Then:
        \begin{align*}
            \begin{split}
                &\mean \left[
                    u_{ m } (x)
                    \mid \data_{ m-1 }
                \right]
                - \nabla \risk ( \hat{ h }_{ m-1 } ) (x) \\
                &\hspace{1.7cm}
                = \mean \left[
                    \hat{ \Phi } ( x, Z ; \data_{ \Phi } )
                    \hat{ \psi } ( Z ; \dataproj )
                    \mid \data_{ m-1 }
                \right]
                - \mean_{ Z } \left[
                    \Phi ( x, Z ) \psi ( Z )
                \right]
            \end{split} \\
            &\hspace{1.7cm}
            = \mean_{ Z } \left[
                \hat{ \Phi } ( x, Z ; \data_{ \Phi } )
                \hat{ \psi } ( Z ; \dataproj )
            \right]
            - \mean_{ Z } \left[
                \Phi ( x, Z ) \psi ( Z )
            \right] \\
            &\hspace{1.7cm}
            = \mean_{ Z } \left[
                \left(
                    \hat{ \Phi } ( x, Z ; \data_{ \Phi } )
                    -
                    \Phi ( x, Z )
                \right) \hat{ \psi } ( Z ; \dataproj )
                +
                \left(
                    \hat{ \psi } ( Z ; \dataproj )
                    -
                    \psi ( Z )
                \right) \Phi ( x, Z )
            \right] \\
            &\hspace{1.7cm}
            = \Bigl<
                \hat{ \Phi } ( x, \cdot ; \data_{ \Phi } )
                - \Phi ( x, \cdot )
                ,
                \hat{ \psi } ( \cdot ; \dataproj )
            \Bigr>_{ L^{ 2 } ( Z ) }
            +
            \Bigl<
                \hat{ \psi } ( \cdot ; \dataproj )
                - \psi ( \cdot )
                ,
                \Phi ( x, \cdot )
            \Bigr>_{ L^{ 2 } ( Z ) } \\
            \begin{split}
                &\hspace{1.7cm}
                \leq \norm{
                    \hat{ \Phi } ( x, \cdot ; \data_{ \Phi } )
                    - \Phi ( x, \cdot )
                }_{ L^{ 2 } ( Z ) }
                \norm{
                    \hat{ \psi } ( \cdot ; \dataproj )
                }_{ L^{ 2 } ( Z ) } \\
                &\hspace{4cm}
                +
                \norm{
                    \hat{ \psi } ( \cdot ; \dataproj )
                    - \psi ( \cdot )
                }_{ L^{ 2 } ( Z ) }
                \norm{
                    \Phi ( x, \cdot )
                }_{ L^{ 2 } ( Z ) }
            \end{split} 
        .\end{align*}
\end{description}

\subsection{Option 2: conditioning on $ \data_{ \Phi, \mathcal{T}, r_{ 0 } } $ and averaging with respect to $ z_{ 1 } , \dots, z_{ M } $}

With this strategy, our aim isn't to get a convergence guarantee analogous to that of the SIP paper, but some sort of ``given $ \dataoff $, this inequality is true with high probability'' guarantee.

We similarly treat each of the three terms separately.

\textbf{First term }
It is bounded in the exact same way as before, since this bound is deterministic and does not involve expectations.

\textbf{Second term }
We are fixing the offline data $ \dataoff $ and averaging with respect to the other samples of the instrumental variable.
Therefore, what we wish to compute is
\begin{align*}
    \mean_{ \bz_{ 1:M } } \left[
        \norm{ u_{ m } }^2
        \given{} \dataoff
    \right]
    &= \mean_{ \bz_{ 1:m } } \left[
        \mean_{ X } \left[
            \hat{ \Phi } ( X, \bz_{ m } )^2
            \partial_{ 2 } \ell \left(
                \hat{ r_{ 0 } } ( \bz_{ m } ),
                \hat{ \mathcal{T} } [ \hat{ h }_{ m-1 } ] ( \bz_{ m } )
            \right)^2
        \right]
        \given{\Big} \dataoff
    \right] \\
    &= \mean_{ X } \left[
        \mean_{ \bz_{ 1:m } } \left[
            \hat{ \Phi } ( X, \bz_{ m } )^2
            \partial_{ 2 } \ell \left(
                \hat{ r_{ 0 } } ( \bz_{ m } ),
                \hat{ \mathcal{T} } [ \hat{ h }_{ m-1 } ] ( \bz_{ m } )
            \right)^2
            \given{\Big} \dataoff
        \right]
    \right]
.\end{align*}
Since $ \bz_{ 1:m } $ is independent from $ \dataoff $, this is equal to
\begin{equation*}
    \mean_{ X } \left[
        \mean_{ \bz_{ 1:m } } \left[
            \hat{ \Phi } ( X, \bz_{ m } )^2
            \partial_{ 2 } \ell \left(
                \hat{ r_{ 0 } } ( \bz_{ m } ),
                \hat{ \mathcal{T} } [ \hat{ h }_{ m-1 } ] ( \bz_{ m } )
            \right)^2
        \right]
    \right]
.\end{equation*}
Reversing back the expectations, we get
\begin{align*}
    &\mean_{ \bz_{ 1:m } } \left[
        \mean_{ X } \left[
            \hat{ \Phi } ( X, \bz_{ m } )^2
            \partial_{ 2 } \ell \left(
                \hat{ r_{ 0 } } ( \bz_{ m } ),
                \hat{ \mathcal{T} } [ \hat{ h }_{ m-1 } ] ( \bz_{ m } )
            \right)^2
        \right]
    \right] \\
    &\hspace{1cm}=
    \mean_{ \bz_{ 1:m } } \left[
        \mean_{ X } \left[
            \hat{ \Phi } ( X, \bz_{ m } )^2
        \right]
        \partial_{ 2 } \ell \left(
            \hat{ r_{ 0 } } ( \bz_{ m } ),
            \hat{ \mathcal{T} } [ \hat{ h }_{ m-1 } ] ( \bz_{ m } )
        \right)^2
    \right]
.\end{align*}
Now we use Assumption 14.5.1 in \cite{sugiyama2012}, which states that
\begin{equation*}
    \sup_{ w \in \mathbb{W} } k ( \bw, \bw ) \leq 1
,\end{equation*}
where $ \mathbb{W} = \mathbb{X} \times \mathbb{Z} $, $ \bw = ( \bx, \bz ) $ and $ k : \mathbb{W} \times \mathbb{W} \to \R $ is the kernel corresponding to the RKHS used to estimate $ \Phi $, which we denote by $ \rkhsw $.
This assumption implies
\begin{align*}
    &\hat{ \Phi } ( \bw )
    = \dotprod{ \hat{ \Phi }, k ( \bw, \cdot ) }_{ \rkhsw }
    \leq \norm{ \hat{ \Phi } }_{ \rkhsw } \norm{ k ( \bw, \cdot ) }_{ \rkhsw }
    = \norm{ \hat{ \Phi } }_{ \rkhsw } \sqrt{ \dotprod{ k ( \bw, \cdot ), k ( \bw, \cdot ) } } = \\
    &\hspace{1.6cm}
    = \norm{ \hat{ \Phi } }_{ \rkhsw } \sqrt{ k ( \bw, \bw ) }
    \leq \norm{ \Phi }_{ \rkhsw }
\end{align*}
for all $ \bw \in \mathbb{W} $.
Therefore,
\begin{align*}
    &\mean_{ \bz_{ 1:m } } \left[
        \mean_{ X } \left[
            \hat{ \Phi } ( X, \bz_{ m } )^2
        \right]
        \partial_{ 2 } \ell \left(
            \hat{ r_{ 0 } } ( \bz_{ m } ),
            \hat{ \mathcal{T} } [ \hat{ h }_{ m-1 } ] ( \bz_{ m } )
        \right)^2
    \right] \\
    &\hspace{1cm}\leq
    \mean_{ \bz_{ 1:m } } \left[
        \mean_{ X } \left[
            \norm{ \hat{ \Phi } }_{ \rkhsw }^2
        \right]
        \partial_{ 2 } \ell \left(
            \hat{ r_{ 0 } } ( \bz_{ m } ),
            \hat{ \mathcal{T} } [ \hat{ h }_{ m-1 } ] ( \bz_{ m } )
        \right)^2
    \right] \\
    &\hspace{1cm}=
    \norm{ \hat{ \Phi } }_{ \rkhsw }^2
    \mean_{ \bz_{ 1:m } } \left[
        \partial_{ 2 } \ell \left(
            \hat{ r_{ 0 } } ( \bz_{ m } ),
            \hat{ \mathcal{T} } [ \hat{ h }_{ m-1 } ] ( \bz_{ m } )
        \right)^2
    \right]
.\end{align*}

To bound the expectation, we assume the loss is quadratic and then
\begin{align*}
    &\mean_{ \bz_{ 1:m } } \left[
        \left(
            \hat{ \mathcal{T} } [ \hat{ h }_{ m-1 } ] ( \bz_{ m } )
            - \hat{ r_{ 0 } } ( \bz_{ m } )
        \right)^2
    \right] \\
    \begin{split}
        &\hspace{1cm}
        = \mean_{ \bz_{ 1:m } } \bigg[
            \Big(
                \left(
                    \hat{ \mathcal{T} } [ \hat{ h }_{ m-1 } ] ( \bz_{ m } )
                    - \mathcal{T} [ \hat{ h }_{ m-1 } ] ( \bz_{ m } )
                \right)
                + \left(
                    r_{ 0 } ( \bz_{ m } )
                    - \hat{ r_{ 0 } } ( \bz_{ m } )
                \right) \\
        &\hspace{4cm}
                + \left(
                    \mathcal{T} [ \hat{ h }_{ m-1 } ] ( \bz_{ m } )
                    - r_{ 0 } ( \bz_{ m } )
                \right)
            \Big)^2
        \bigg]
    \end{split} \\
    \begin{split}
        &\hspace{1cm}
        \leq 3 \mean_{ \bz_{ 1:m } } \bigg[
            \left(
                \hat{ \mathcal{T} } [ \hat{ h }_{ m-1 } ] ( \bz_{ m } )
                - \mathcal{T} [ \hat{ h }_{ m-1 } ] ( \bz_{ m } )
            \right)^2
            + \left(
                r_{ 0 } ( \bz_{ m } )
                - \hat{ r_{ 0 } } ( \bz_{ m } )
            \right)^2 \\
    &\hspace{3cm}
            + \left(
                \mathcal{T} [ \hat{ h }_{ m-1 } ] ( \bz_{ m } )
                - r_{ 0 } ( \bz_{ m } )
            \right)^2
        \bigg]
    \end{split} \\
    \begin{split}
        &\hspace{1cm}
        = 3 \bigg\{
            \mean_{ \bz_{ 1:m-1 } }\left[
                \norm{
                ( \hat{ \mathcal{T} } - \mathcal{T} )
                [ \hat{ h }_{ m-1 } ]
            }_{ L^2 ( \mathbb{Z} ) }^2
            \right]
            + \mean_{ \bz_{ 1:m } } \left[
                \norm{ r_{ 0 } - \hat{ r_{ 0 } } }_{ L^2 ( \mathbb{Z} ) }^2
            \right]\\
        &\hspace{3cm}
            + \mean_{ \bz_{ 1:m-1 } }  \left[
                \norm{ \mathcal{T} [ \hat{ h }_{ m-1 } ] - r_{ 0 } }_{ L^2 ( \mathbb{Z} ) }^2
            \right]
        \bigg\}
    \end{split}
.\end{align*}
We treat each part of this expression separately.
Firstly,
\begin{align*}
    \norm{ ( \hat{ \mathcal{T} } - \mathcal{T} ) [ \hat{ h }_{ m-1 } ] }_{ L^2 ( \mathbb{Z} ) }^2
    \leq \norm{ \hat{ \mathcal{T} } - \mathcal{T} }_{ \mathrm{op} }^2 \norm{ \hat{ h }_{ m-1 } }_{ L^2 ( \mathbb{X} ) }^2
    \leq M^2 \norm{ \hat{ \mathcal{T} } - \mathcal{T} }_{ \mathrm{op} }^2
.\end{align*}
We leave the second term as $ \norm{ r_{ 0 } - \hat{ r_{ 0 } } }_{ L^{ 2 } ( \mathbb{Z} ) }^2 $.
Finally, for the third term, we have
\begin{align*}
    \norm{ \mathcal{T} [ \hat{ h }_{ m-1 } ] - r_{ 0 } }_{ L^2 ( \mathbb{Z} ) }^2
    &= \mean_{ Z } \left[
        \left(
            \mathcal{T} [ \hat{ h }_{ m-1 } ] ( Z ) - r_{ 0 } ( Z )
        \right)^2
    \right] \\
    &= \mean_{ Z } \left[
        \left(
            \mean \left[ \hat{ h }_{ m-1 } ( X ) - Y \given{} Z \right]
        \right)^2
    \right] \\
    &\leq \mean_{ ( X, Y ) } \left[
        \left(
            \hat{ h }_{ m-1 } ( X ) - Y
        \right)^2
    \right] \\
    &\leq 2 \left(
        \mean_{ X } \left[ \hat{ h }_{ m-1 } ( X )^2 \right]
        + \mean \left[ Y^2 \right]
    \right) \\
    &= 2 \left(
        \norm{ \hat{ h }_{ m-1 } }_{ L^2 ( \mathbb{X} ) }^2
        + \mean \left[ Y^2 \right]
    \right) \\
    &\leq 2 \left( M^2 + \mean \left[ Y^2 \right] \right)
.\end{align*}
Putting everything together, what we conclude is
\begin{align*}
    \mean_{ \bz_{ 1:m } } \left[
        \norm{ u_{ m } }_{ L^2 ( \mathbb{X} ) }^2 \given{} \dataoff
    \right]
    \leq
    3 \norm{ \hat{ \Phi } }_{ \rkhsw }^2 \left(
        M^2 \norm{ \hat{ \mathcal{T} } - \mathcal{T} }_{ \mathrm{op} }^2
        + \norm{ r_{ 0 } - \hat{ r_{ 0 } } }_{ L^2 ( \mathbb{Z} ) }^2
        + 2 \left( M^2 + \mean [ Y^2 ] \right)
    \right)
.\end{align*}

% \begin{align*}
%     &\norm{ \hat{ \Phi } }_{ \rkhsw }^2
%     \mean_{ \bz_{ 1:m } } \left[
%         \left(
%             \hat{ \mathcal{T} } [ \hat{ h }_{ m-1 } ] ( \bz_{ m } )
%             - \hat{ r_{ 0 } } ( \bz_{ m } )
%         \right)^2
%     \right] \\
%     \begin{split}
%         &\hspace{1cm}
%         = \mean_{ \bz_{ 1:m } } \bigg[
%             \mean_{ X } \left[
%                 \hat{ \Phi } ( X, \bz_{ m } )^2
%             \right]
%             \cdot \Big(
%                 \left(
%                     \hat{ \mathcal{T} } [ \hat{ h }_{ m-1 } ] ( \bz_{ m } )
%                     - \mathcal{T} [ \hat{ h }_{ m-1 } ] ( \bz_{ m } )
%                 \right) \\
%                 &\hspace{5.5cm}
%                 + \left(
%                     r_{ 0 } ( \bz_{ m } )
%                     - \hat{ r_{ 0 } } ( \bz_{ m } )
%                 \right) \\
%                 &\hspace{5.5cm}
%                 + \left(
%                     \mathcal{T} [ \hat{ h }_{ m-1 } ] ( \bz_{ m } )
%                     - r_{ 0 } ( \bz_{ m } )
%                 \right)
%             \Big)^2
%         \bigg]
%     \end{split} \\
%     \begin{split}
%         &\hspace{1cm}
%         \leq 3 \mean_{ \bz_{ 1:m } } \bigg[
%             \mean_{ X } \left[
%                 \hat{ \Phi } ( X, \bz_{ m } )^2
%             \right]
%             \left(
%                 \hat{ \mathcal{T} } [ \hat{ h }_{ m-1 } ] ( \bz_{ m } )
%                 - \mathcal{T} [ \hat{ h }_{ m-1 } ] ( \bz_{ m } )
%             \right)^2
%             \\
%             &\hspace{2.75cm}
%             + \mean_{ X } \left[
%                 \hat{ \Phi } ( X, \bz_{ m } )^2
%             \right]
%             \left(
%                 r_{ 0 } ( \bz_{ m } )
%                 - \hat{ r_{ 0 } } ( \bz_{ m } )
%             \right)^2
%             \\
%             &\hspace{2.75cm}
%             + \mean_{ X } \left[
%                 \hat{ \Phi } ( X, \bz_{ m } )^2
%             \right]
%             \left(
%                 \mathcal{T} [ \hat{ h }_{ m-1 } ] ( \bz_{ m } )
%                 - r_{ 0 } ( \bz_{ m } )
%             \right)^2
%         \bigg]
%     \end{split}
% .\end{align*}
% 
% {\color{red} Don't know what else to do with this for now.
% My guess is well have to use some boundedness assumptions on the RKHSs used to estimate $ \Phi, \mathcal{T} $ and $ r_{ 0 } $ to uniformly bound this expression.
% The goal is to bound this uniformly with respect to $ \bz_{ 1 }, \bz_{ 2 }, \dots $ and in a way which does not explode as $ \abs{ \dataoff } $ and $ M $ grow.}


\textbf{Third term }
