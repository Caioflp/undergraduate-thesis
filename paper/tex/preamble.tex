% if you need to pass options to natbib, use, e.g.:
%     \PassOptionsToPackage{numbers, compress}{natbib}
% before loading neurips_2023


% ready for submission
\usepackage[nonatbib]{neurips_2023}
\usepackage{global-macros}


% to compile a preprint version, e.g., for submission to arXiv, add add the
% [preprint] option:
%     \usepackage[preprint]{neurips_2023}

% to compile a camera-ready version, add the [final] option, e.g.:
%     \usepackage[final]{neurips_2023}


% to avoid loading the natbib package, add option nonatbib:
%    \usepackage[nonatbib]{neurips_2023}


% \usepackage[utf8]{inputenc} % allow utf-8 input
\usepackage[T1]{fontenc}    % use 8-bit T1 fonts
\newcommand{\theHalgorithm}{\arabic{algorithm}}
\usepackage[algoruled,boxed,lined]{algorithm2e}
\usepackage{hyperref}       % hyperlinks
\usepackage{url}            % simple URL typesetting
\usepackage{booktabs}       % professional-quality tables
\usepackage{amsfonts}       % blackboard math symbols
\usepackage{nicefrac}       % compact symbols for 1/2, etc.
\usepackage{microtype}      % microtypography
\usepackage{enumitem}
\usepackage{bbm}
\usepackage[dvipsnames]{xcolor}         % colors
\usepackage{biblatex}
\addbibresource{../bib/bibliography.bib}

\usepackage{xargs}                      % Use more than one optional parameter in a new commands
% \usepackage[pdftex,dvipsnames]{xcolor}  % Coloured text etc.

\usepackage[colorinlistoftodos,prependcaption,textsize=tiny]{todonotes}
\newcommandx{\unsure}[2][1=]{\todo[linecolor=red,backgroundcolor=red!25,bordercolor=red,#1]{#2}}
\newcommandx{\change}[2][1=]{\todo[linecolor=blue,backgroundcolor=blue!25,bordercolor=blue,#1]{#2}}
\newcommandx{\info}[2][1=]{\todo[linecolor=OliveGreen,backgroundcolor=OliveGreen!25,bordercolor=OliveGreen,#1]{#2}}
\newcommandx{\improvement}[2][1=]{\todo[linecolor=Plum,backgroundcolor=Plum!25,bordercolor=Plum,#1]{#2}}
\newcommandx{\thiswillnotshow}[2][1=]{\todo[disable,#1]{#2}}


\title{Stochastic Gradient Descent in NPIV estimation}


% The \author macro works with any number of authors. There are two commands
% used to separate the names and addresses of multiple authors: \And and \AND.
%
% Using \And between authors leaves it to LaTeX to determine where to break the
% lines. Using \AND forces a line break at that point. So, if LaTeX puts 3 of 4
% authors names on the first line, and the last on the second line, try using
% \AND instead of \And before the third author name.


\author{%
}

\newcommand{\hstar}{h^{ \star }}
\newcommand{\meanop}{\mathcal{P}}
\newcommand{\op}{\mathrm{op}}
\newcommand{\risk}{\mathcal{R}}
\newcommand{\loss}{\ell}
\renewcommand{\hat}{\widehat}
\newcommand{\iid}{\overset{\mathrm{iid}}{\sim}}
\DeclareMathOperator{\diam}{diam}
\newcommand{\data}{\mathcal{D}}
\newcommand{\dataoff}{\mathcal{D}_{ \Phi, \meanop, r_{ 0 } }}
\newcommand{\dataproj}{\mathcal{D}_{ \mathrm{proj} }}
% \newcommand{\ind}{\mathbbm{1}}
\newcommand{\rkhsw}{\mathcal{R}_{ \mathbb{W} }}

% Environments
\newtheorem{assumption}{Assumption}
\newtheorem{proposition}{Proposition}
\newtheorem{lemma}{Lemma}
\newtheorem{theorem}{Theorem}
\DeclareMathOperator{\ber}{Bernoulli}
\DeclareMathOperator{\BCE}{BCE}

\hfuzz=15pt
