\section{Gradient computation}

We'd like to compute $ \nabla \risk ( h ) $ for $ h \in L^{ 2 } ( X ) $.
We start by computing the directional derivative of $ \risk $ at $ h $ in the direction $ f $, denoted by $ D \risk [ h ] ( f ) $:
\begin{align*}
    D \risk [ h ] ( f )
    &= \lim\limits_{ \delta \to 0 } \frac{ 1 }{ \delta } \left[
        \risk ( h + \delta f ) - \risk ( f )
    \right] \\
    &= \lim\limits_{ \delta \to 0 } \frac{ 1 }{ \delta } \mean \left[
        \ell ( r_{ 0 } ( Z ), \mathcal{T} [ h + \delta f ] ( Z ) )
        -
        \ell ( r_{ 0 } ( Z ), \mathcal{T} [ h ] ( Z ) )
    \right] \\
    &= \lim\limits_{ \delta \to 0 } \frac{ 1 }{ \delta } \mean \left[
        \ell ( r_{ 0 } ( Z ), \mathcal{T} [ h ] ( Z ) + \delta \mathcal{T} [ f ] ( Z ) )
        -
        \ell ( r_{ 0 } ( Z ), \mathcal{T} [ h ] ( Z ) )
    \right] \\
    \begin{split}
        &= \lim\limits_{ \delta \to 0 } \frac{ 1 }{ \delta } \mean \Biggl[
            \delta \partial_{ 2 } \ell ( r_{ 0 } ( Z ), \mathcal{T} [ h ] ( Z ) ) \mathcal{T} [ f ] ( Z ) \\
        &\hspace{2cm}+ \frac{ \delta^2 }{ 2 } \partial_{ 2 }^2 \ell ( r_{ 0 } ( Z ) , \mathcal{T} [ h + \theta f ] ( Z ) ) \mathcal{T} [ f ] ( Z )^2
        \Biggr]
    \end{split} \\
    \begin{split}
        &= \mean \left[
            \partial_{ 2 } \ell ( r_{ 0 } ( Z ), \mathcal{T} [ h ] ( Z ) ) \mathcal{T} [ f ] ( Z )
        \right] \\
        &\hspace{2cm}+ \lim\limits_{ \delta \to 0 } \mean \Biggl[
            \frac{ \delta }{ 2 } \partial_{ 2 }^2 \ell ( r_{ 0 } ( Z ) , \mathcal{T} [ h + \theta f ] ( Z ) ) \mathcal{T} [ f ] ( Z )^2
        \Biggr]
    \end{split} \\
    &= \mean \left[
        \partial_{ 2 } \ell ( r_{ 0 } ( Z ), \mathcal{T} [ h ] ( Z ) ) \mathcal{T} [ f ] ( Z )
    \right]
,\end{align*}
where $ \theta \in \R $ is due to Taylor's formula and can be assumed\info{Assumption} to be inside a fixed interval $ ( -\theta_{ 0 }, \theta_{ 0 } ) $, with $ \theta_{ 0 } $ arbitrarily small.
We have assumed\info{Assumption} at the last step that there exists $ \theta_{ 0 } > 0 $ such that
\begin{equation*}
    \sup_{ \abs{ \theta } < \theta_{ 0 } }
    \mean \left[
        \partial_{ 2 }^2 \ell ( r_{ 0 } ( Z ), \mathcal{T} [ h + \theta f ] ( Z ) ) \mathcal{T} [ f ] ( Z )^2
    \right] < \infty
.\end{equation*}
This is a mild integrability condition which can be shown to hold in the quadratic case.

We can in fact expand the calculation a bit more, as follows:
\begin{align*}
    D \risk [ h ] ( f )
    &= \mean \left[
        \partial_{ 2 } \ell ( r_{ 0 } ( Z ), \mathcal{T} [ h ] ( Z ) ) \mathcal{T} [ f ] ( Z )
    \right] \\
    &= \dotprod{ \partial_{ 2 } \ell ( r_{ 0 } ( \cdot ), \mathcal{T} [ h ] ( \cdot ) ), \mathcal{T} [ f ] }_{ L^{ 2 } ( Z ) } \\
    &= \dotprod{ \mathcal{T}^{ * } [ \partial_{ 2 } \ell ( r_{ 0 } ( Z ), \mathcal{T} [ h ] ( \cdot ) ) ], f }_{ L^{ 2 } ( X ) }
,\end{align*}
where we are assuming\info{Assumption} that $ \partial_{ 2 } \ell ( r_{ 0 } ( \cdot ), \mathcal{T} [ h ] ( \cdot ) ) \in L^{ 2 } ( Z ) $.
This shows that $ \risk $ is Gateux-differentiable, with Gateux derivative at $ h $ given by
\begin{equation*}
    D \risk [ h ] = \mathcal{T}^{ * } [ \partial_{ 2 } \ell ( r_{ 0 } ( \cdot ), \mathcal{T} [ h ] ( \cdot ) ) ]
.\end{equation*}
If we assume\footnote{It is if $ \ell $ is quadratic.}\info{Assumption} that $ h \mapsto D \risk [ h ] $ is a continuous mapping from $ L^{ 2 } ( Z ) $ to $ L^{ 2 } ( Z ) $\improvement{Talk about which conditions $ \ell $ can satisfy so that this is continuous.}, then $ \risk $ is also Fréchet-differentiable, and both derivatives coincide.
Therefore, under this assumption, which we henceforth make, $ \nabla \risk ( h ) = \mathcal{T}^{ * } [ \partial_{ 2 } \ell ( r_{ 0 } ( \cdot ), \mathcal{T} [ h ] ( \cdot ) ) ] $.
